\section{Kategorisierung anhand technischer Grundlage}
Kapitel \ref{ch:methoden} gliedert Methoden zur Sicherung der Vertraulichkeit in den Lebenszyklus eines Modells, von der Vorbereitung der Daten bis zum Betrieb des Modells, ein. 
Eine alternative Kategorisierung erfolgt anhand der technischen und mathematischen Grundlage der Methoden.
Hier ergeben sich zwei Hauptkategorien: statistische Methoden und kryptografische Methoden.

Kryptografische Methoden, wie der Name bereits andeutet, nutzen Kryptografie, um Daten oder Berechnungen auf diesen Daten zu verschlüsseln.
Die hier genutzten Basistechniken sind Homomorphe Verschlüsselung, Funktionale Verschlüsselung und andere Secure Multi-Party Computation Methoden wie Garbled Circuits.
Diese ermöglichen es, den Trainingsprozess und die Vorhersageberechnung des Modells auf verschlüsselten Daten durchzuführen.
Kapitel \ref{sec:bw_krypto} zeigt, ob und wie diese Methoden sinnvoll genutzt werden können.

Statistische Methoden beeinflussen Daten, oder Berechnungen mit diesen, sodass die Vertraulichkeit dieser geschützt werden kann.
Anonymisierung ist beispielsweise eine Technik, welche durch Gruppierungen von Merkmalen dafür sorgen kann, dass einzelne Datenpunkte nicht eindeutig zugeordnet werden können.
Die Technik, die jedoch aktuell im Kontext Machine Learning am populärsten ist, ist Differential Privacy. 
Unternehmen wie Apple \cite{apple_dp}, Google \cite{google_dp},  Meta \cite{meta_dp} und Snapchat \cite{snapchat_dp} nutzen Differential Privacy bereits in Kombination mit Neuronalen Netzen.
Aus diesem Grund wird Differential Privacy als Vertreter für statistische Methoden in Kapitel \ref{sec:bw_dp} bewertet.