\section{Bewertung kryptografischer Methoden}\label{sec:bw_krypto}

Moderne kryptografische Techniken ermöglichen es, Berechnungen auf verschlüsselten Daten durchzuführen.
Im Folgenden werden diese Methoden anhand von Beispielen, je eine Methode pro kryptografische Technik, analysiert.
Dabei wird betrachtet, an welcher Stelle die Methoden angewendet werden können, wie die Rechenleistung und Güte der Modelle beeinflusst wird und ob die Methoden Vertraulichkeit schützen können.

Tabelle \ref{tab:krypto_methods} zeigt die bereits vorgestellten kryptografischen Methoden, welche auf Neuronale Netze angewendet werden können.
Diese sind nach anhand der zugrundeliegenden Kryptografie gruppiert.
Die Phase des Modells beschreibt dabei, ob die Methoden das Training oder nur die Inferenz unterstützen.

\begin{table}[!htb]
\centering
\begin{tabular}{|l|l|l|}
\hline
\rowcolor[HTML]{CBCEFB} 
Art der Kryptografie                                                                     & Phase des Modells                                              & Methode                                                                                                                         \\ \hline
                                                                                         & \begin{tabular}[c]{@{}l@{}}Training +\\ Inferenz\end{tabular}  & \begin{tabular}[c]{@{}l@{}}Eingeschränkte homomorphe \\ Verschlüsselung nach \\ Takabi et al. \cite{P-104}\end{tabular}                \\ \cline{2-3} 
                                                                                         & Inferenz                                                       & CryptoNets \cite{P-54}                                                                                                                 \\ \cline{2-3} 
\multirow{-3}{*}{\begin{tabular}[c]{@{}l@{}}Homomorphe \\ Verschlüsselung\end{tabular}}  & Inferenz                                                       & \begin{tabular}[c]{@{}l@{}}Eingeschränkt homomorphe \\ Verschlüsselung nach \\ Chabanne et al. \cite{P-55}\end{tabular}                \\ \hline
                                                                                         & \begin{tabular}[c]{@{}l@{}}Training + \\ Inferenz\end{tabular} & CryptoNN \cite{P-53}                                                                                                                   \\ \cline{2-3} 
                                                                                         & Inferenz                                                       & \begin{tabular}[c]{@{}l@{}}Funktionale Verschlüsselung \\ für Polynome mit Grad 2 \\ nach Dufour-Sans et al. \cite{P-105}\end{tabular} \\ \cline{2-3} 
\multirow{-3}{*}{\begin{tabular}[c]{@{}l@{}}Funktionale \\ Verschlüsselung\end{tabular}} & Inferenz                                                       & \begin{tabular}[c]{@{}l@{}}Funktionale Verschlüsselung \\ für Polynome mit Grad 2 \\ nach Ryffel et al. \cite{P-46}\end{tabular}       \\ \hline
                                                                                         & Inferenz                                                       & DeepSecure \cite{P-71}                                                                                                                 \\ \cline{2-3} 
                                                                                         & Inferenz                                                       & Chameleon \cite{P-72}                                                                                                                  \\ \cline{2-3} 
\multirow{-3}{*}{Yao's Garbled Circuits}                                                 & Inferenz                                                       & MiniONN \cite{P-59}                                                                                                                    \\ \hline
\end{tabular}
\caption{Übersicht kryptografischer Methoden}
\label{tab:krypto_methods}
\end{table}


\subsubsection*{Phase des Modells}

Jede der dargestellten Methoden unterstützt die Inferenz, doch nicht zwingend den Trainingsprozess selbst.
Dies liegt daran, dass jeder der kryptografischen Methoden dafür sorgt, dass Berechnungen um ein Vielfaches komplexer werden. 
Ein Schritt des Trainingsprozesses ist wesentlich komplexer als die Vorhersage. 
Dies liegt unter anderem daran, dass dieser eine Inferenz enthält.
Zusätzlich muss für jedes Gewicht ein Gradient anhand der Verlustfunktion berechnet werden, anhand dessen jedes Gewicht angepasst wird.
Eine Nutzung von Kryptografie an dieser Stelle würde einen wesentlich höheren Mehraufwand, sowohl von der Komplexität als auch der Performance, bedeuten.
Dieser ist sogar so hoch, dass es (mit aktuellen Techniken) nicht sinnvoll wäre, das Training kryptografisch zu sichern.
Dies ist der Grund, warum sich kryptografische Methoden auf die Inferenz des Modells fokussieren.

\subsubsection*{Performance}

Da kryptografische Methoden zusätzliche Schritte wie eine Ver- und Entschlüsselung beinhalten, ist es nachvollziehbar, dass Mehraufwand bei Berechnungen entsteht.
Jedoch ist dieser verhältnismäßig hoch.

Ein Beispiel für homomorphe Verschlüsselung wäre hier CryptoNets \cite{P-54}.
Zur Evaluation wird ein relativ einfaches Neuronales Netz genutzt, welches aus zwei Faltungsschichten, gefolgt von je einer Aktivierungsfunktion und Pooling Schicht, sowie zwei vollständig verbundenen Schichten besteht.
Die Autoren zeigen zwar, dass eine Server-CPU in der Lage ist, bei Vollauslastung fast 60.000 Vorhersagen zu treffen, jedoch berücksichtigt dies nicht die Schritte der Ver- und Entschlüsselung.
Die gleiche CPU ist in der Lage, 25.000 Datensätze (Bilder) pro Stunde zu verschlüsseln und 275.000 Ergebnisse zu entschlüsseln.
Würde man alle Schritte kombinieren, benötigen die 60.000 Vorhersagen ungefähr 2 Stunden und 25 Minuten für die Verschlüsselung und 15 Minuten für die Entschlüsselung zusätzlich.
Somit erhöht sich die Gesamtzeit auf 3 Stunden und 40 Minuten.
Da ein Modell oftmals nicht unter Vollauslastung läuft, ist auch die Inferenzzeit eines einzelnen Datensatzes relevant.
Die Verschlüsselung benötigt 44,5 Sekunde, die Vorhersage des Modells 250 Sekunden und die Entschlüsselung des Ergebnisses 3 Sekunden. 
Folglich dauert die Inferenz eines Datensatzes insgesamt 297,5 Sekunden.

Da die Autoren nicht spezifizieren, ob CryptoNets durch Grafikkarten (GPUs) beschleunigt werden könnte, wird ebenfalls mit einer CPU verglichen.
Dabei wird ein vergleichbares Modell mit dem Framework PyTorch auf einer Desktop-CPU nachgebaut.
Es ist anzumerken, dass die genutzte Desktop-CPU im Vergleich zu der von CryptoNets genutzten Server-CPU 6 Jahre neuer ist, 2 Kerne mehr hat und eine etwas höhere Taktrate besitzt.
Die Differenz der Ergebnisse wären bei gleicher Hardware geringer, jedoch sind die Ergebnisse dennoch aussagekräftig.
Die Klassifikation von 60.000 Datensätzen benötigt weniger als eine Minute und ein einzelnes Bild kann in weniger als einer Sekunden inferiert werden.
Dementsprechend benötigt CryptoNets bei größeren Datenmengen unter Vollauslastung 150 Mal länger als ein unverschlüsseltes Modell. 
Bei einem einzigen Datensatz entspricht der Faktor sogar 300.
Bei größeren Modellen würde sich der Faktor sogar erhöhen, was bedeutet, dass die kryptografische Inferenz mit steigender Modellgröße immer unpraktikabler wird.


CryptoNN \cite{P-53}, ein auf funktionaler Verschlüsselung basierendes Framework, bietet eine bessere Performance.
Dies liegt daran, dass nicht das ganze Modell kryptografisch berechnet wird, sondern nur die erste Schicht des Modells.
Der dadurch reduzierte Schutz des Modells wird im Laufe des Kapitels genauer beleuchtet.
Zur Evaluation der Trainingszeit nutzten die Autoren LeNet-5 Modell, ein simples Neuronales Netz mit zwei Faltungsschichten, welches mit der MNIST Datenmenge\cite{D-MNIST} und zwei Epochen trainiert wurde.
Unverschlüsselt dauert das Training (auf der Hardware der Autoren) 4 Stunden und verschlüsselt 57 Stunden, somit über 14 Mal länger.
Die Autoren geben keine Zeit für die reine Inferenz an, deshalb wird angenommen, dass der Multiplikationsfaktor der Zeit mindestens gleich groß ist.

Mit dem Chameleon Framework \cite{P-72} wird das ganze Modell verschlüsselt von zwei Parteien mittels Oblivious Transfer und Garbled Circuits inferiert.
Bei dem gleichen Modell, welches auch CrytoNets zur Evaluation nutzt, zeigen die Autoren, dass die Klassifikation eines Datensatzes 3 Sekunden dauert.
Jedoch ist der Leistungsvorteil, wenn Daten gebündelt als Batch klassifiziert werden, kleiner als bei CryptoNN.
Dies sorgt dafür, dass die Klassifikation von 60.000 Bildern, mit einer Batch Größe von 100, ungefähr 26 Stunden benötigen würde.
Im Vergleich zum unverschlüsselten Modell mit PyTorch, benötigt die Klassifikation eines einzelnen Datensatzes 3 Mal mehr, von 60.000 Bilder jedoch 1560 Mal mehr.
Zusätzlich ist das Chameleon auf eine schnelle und stabile Netzwerkverbindung angewiesen, was den Leistungunterschied zu den normalen Modellen noch erhöhen könnte.

\subsubsection*{Qualität der Modelle}

Neben dem Mehraufwand der Berechnungen, gibt es zusätzlich Einschränkungen bei der Erstellung des Modells, was auch die Güte des Modells beeinflussen könnte.

Um Leistung zu sparen, setzt CryptoNets \cite{P-54} auf eine eingeschränkte homomorphe Verschlüsselung, welche nur eine begrenzte Anzahl an Operationen ermöglicht.
CryptoNets ermöglicht nur Polynomberechnungen, was dafür sorgt, dass Funktionen, die sich nicht als Polynom abbilden lassen, nur approximiert werden können.
Dazu zählen Pooling Schichten, aber auch Aktivierungsfunktionen wie ReLU.
Bei dem Modell, welches oben beschrieben ist, erreichen die Autoren eine Genauigkeit von knapp 99\%.
Somit hat die Approximation, zumindest bei kleinen Neuronalen Netzen, kaum Auswirkungen auf die Güte.
Da CryptoNN \cite{P-53} nur die erste Schicht eines Modells verschlüsselt, muss das Framework nicht jede Art von Schicht unterstützen.
Faltungsschichten und vollständig verbunden Schichten werden unterstützt, wodurch die Genauigkeit eines Modells nicht negativ beeinflusst wird.
Das Framework Chameleon \cite{P-72} hat ebenfalls keinen negativen Einfluss auf die Genauigkeit von Modellen.
Das gleiche Modell wie bei CryptoNets, erreicht mit Chameleon ebenfalls eine Genauigkeit von 99\%.

\subsubsection*{Schutz der Vertraulichkeit}

Public Clouds ermöglichen es, moderne und schnelle Hardware nach Belieben zu buchen und zu nutzen.
Dies schließt ebenfalls GPUs ein, welche für das Training und die Vorhersage Neuronaler Netze optimiert ist.
Ein Problem, welches dabei entsteht ist, dass Cloud Provider theoretisch in der Lage wären, Daten auf den Servern mitzulesen.
Homomorphe Verschlüsselung könnte dieses Problem lösen, da sich weder die Eingabedaten, noch die Vorhersage eines Modells unverschlüsselt auf dem Server befinden.
Der Modellbetreiber wäre ebenfalls nicht in der Lage, diese Daten einzusehen.

Da bei der funktionalen Verschlüsselung das Ergebnis jeder Schicht im Klartext vorliegt, bedeutet dies folgend, dass das Label der Vorhersage für den Serverbetreiber und den Modellbetreiber erkennbar sind.
Das Modell, beziehungsweise die Anwendung um das Modell herum, kann so implementiert werden, dass der Serverbetreiber beispielsweise die Labels der Vorhersage nicht zuordnen können.
Der Modellbetreiber hingegen könnte sogar das Ergebnis der Vorhersage mit Metadaten der Anfrage kombinieren und so dennoch Informationen über den Nutzer des Modells erhalten. 
Soll ein Modell beispielsweise anhand eines Bildes eine medizinische Diagnose vorhersagen, kann der Modellbetreiber erfahren, welcher Nutzer welche Krankheit hat, ohne jedoch das Bild zu kennen.
Bei Garbled Circuits kann der Modellbetreiber ebenfalls das Label auslesen, was zu den gleichen Problemen führt, die auch funktionale Verschlüsselung hat.

Ein Vorteil der kryptografischen Methoden ist die mathematische Beweisbarkeit.
Diese ermöglicht es, den Schutz gegen Angriffe zu quantifizieren, indem die Anzahl an Schritten angegeben werden kann, die ein Angreifer ausführen müsste, um die Verschlüsselung zu knacken.
Zusätzlich kann die Schlüssellänge erhöht werden, sodass bei Bedarf die Sicherheit auch angepasst ist.