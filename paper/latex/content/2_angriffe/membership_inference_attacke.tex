\section{Membership Inference Angriff}\label{sec:membership_inference}

Bei der Membership Inference Attacke versucht ein Angreifer herauszufinden, ob ein Datenpunkt Bestandteil des Trainingsdatensatzes eines Modells ist. 
Dies bedroht die Vertraulichkeit, da beispielsweise herausgefunden werden kann, ob eine bestimmte Person Teil eines Trainingsdatensatzes für die Diagnose einer Krankheit ist und folglich auch mit der entsprechenden Krankheit diagnostiziert ist \cite{P-2}.

Shokri et al. \cite{P-2} führen eine Membership Inference Attacke durch, indem eine Reihe Modelle trainiert wird, die ähnlich dem Angriffsziel-Modell sind.
Diese Modelle werden auch Shadow Modelle genannt.
Ähnlich bedeutet hier, dass sowohl die Funktion der Modelle, als auch der Trainingsdatensatz, mit dem angegriffenen Modell vergleichbar sind.
Einige dieser trainierten Modelle enthalten den zu untersuchenden Datenpunkt im Trainingsdatensatz, andere hingegen nicht.
Nun wird ein binärer Meta-Klassifikator trainiert (analog zu der Property Inference Attacke in Kapitel \ref{sec:property_inference}), welcher anhand der Vorhersagen der Shadow Modelle, Label und Confidence Score, lernt, ob ein Datensatz im Training des entsprechenden Modells genutzt wurde.
Wird in diesen Meta-Klassifikator die Vorhersage des angegriffenen Modells als Input genutzt, so erhält man die Antwort, ob der Datenpunkt für das Modelltraining genutzt wurde oder nicht.
Laut Shokri et al. \cite{P-2} funktioniert dieser Angriff, da ähnliche Modelle, die mit einem ähnlichen Datensatz trainiert wurden, sich auch ähnlich verhalten. 
Somit kann bei den selbst trainierten Modellen, welche den Datensatz enthalten, ein Muster gelernt werden, welches auch auf andere Modelle anwendbar ist. 
Overfitting und eine komplexe Modellarchitektur erhöhen die Wahrscheinlichkeit eines erfolgreichen Angriffs.

Carlini et al. \cite{P-13} zeigen eine Alternative des Angriffs, bei welcher kein Meta-Klassifikator genutzt wurde.
Die Shadow Modelle werden analog zu \cite{P-2} trainiert. 
Anstatt mit den Vorhersagen dieser Modelle nun einen Meta-Klassifikator zu trainieren, werden zwei Gaußverteilungen gebildet, jeweils über die Confidence Scores der Modelle, wo der Datenpunkt im Training enthalten war oder nicht.
Mittels eines Likelihood-Quotienten-Tests wird anschließend vorhergesagt, in welcher Verteilung der Confidence Score des angegriffenen Modells wahrscheinlicher liegt.
Ein Vorteil dieser Methode liegt darin, dass weniger Shadow Modelle trainiert werden müssen, da bereits mit relativ wenig Werten eine Gaußverteilung modelliert werden kann.


Eine Voraussetzung bei Shokri et al. \cite{P-2} ist es, dass der Confidence Score mit ausgegeben wird.
Choquette-Choo et al. \cite{P-7} wandeln den Angriff ab, sodass dieser Score nicht mehr benötigt wird.
Der Angriff funktioniert simultan zu \cite{P-2}, jedoch wird nicht nur der Datenpunkt selber in die Modelle als Input gegeben, sondern auch Abwandlung davon. 
Diese Abwandlungen könnte das Hinzufügen von zufälligem Rauschen sein, oder bei Bilddateien beispielsweise Rotation oder Translation.
Die Hypothese der Autoren ist, dass das Modell bei Datenpunkten, die im Training genutzt wurden, robuster gegenüber diesen Abwandlungen ist und dennoch den Datenpunkt korrekt klassifiziert.
Zusätzlich könnten Abwandlungen der Daten über einen Data Augmentation Schritt auch direkt vom Modell gelernt worden sein.

