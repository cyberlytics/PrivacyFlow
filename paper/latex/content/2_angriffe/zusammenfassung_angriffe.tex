\section{Zusammenfassung der Angriffe}

Ein Datenleck ist eines der größten Probleme von datengetriebenen Anwendungen.
Selbst wenn diese anonymisiert wurden, können diese oftmals zu den Originaldaten zurückgeführt werden.
Dies erfolgt oftmals statistisch mit einem weiteren (teils öffentlichem) Datensatz (Kapitel \ref{sec:deano}).

Jedoch können Daten auch rekonstruiert werden, wenn nur das Modell bereitgestellt wird.
Dies wird Model Inversion Attacke genannt.
Dabei versucht ein Angreifer, die Vorhersage eines Modells zu nutzen, um seinen Datenpunkt näher an den Originaldatenpunkt zu bringen.
Dies kann über einen iterativen Prozess funktionieren oder auch komplexere, generative Modelle wie Autoencoder nutzen (Kapitel \ref{sec:model_inversion}).

Wenn das Ziel nicht einzelne Datenpunkte sind, sondern Eigenschaften, die für einen ganzen Datensatz gelten, können diese mit der Property Inference Attacke analysiert werden.
Dabei werden Shadow Modelle trainiert, welche versuchen, das originale Modell zu imitieren.
Das Besondere an diesen Shadow Modellen ist es, dass manche mit einem Datensatz trainiert wurden, der die zu untersuchende Eigenschaft besitzt und andere wiederum nicht.
Die Shadow Modelle werden genutzt, um einen Klassifikator zu bauen, der bestimmt, ob die Eigenschaft im genutzten Datensatz vorkommt oder nicht.
Wird dies auf das originale Modell angewendet, kann ein Angreifer Aussagen zu dieser Eigenschaft treffen (Kapitel \ref{sec:property_inference}).

Die Membership Inference Attacke nutzt ebenfalls Shadow Modelle, um einen Klassifikator zu trainieren, jedoch diesmal um herauszufinden, ob ein spezifischer Datenpunkt im Training genutzt wurde.
Alleine dieser Fakt könnte sensibel sein (Kapitel \ref{sec:membership_inference}).

Sprachmodelle, welche das nächste Wort vorhersagen, sind besonders von der Data Extraction Attacke betroffen.
Durch lange, fortlaufende Nutzung des Modells wurde ein riesiger Datensatz aus Text gesammelt, welcher nur vom Modell stammt.
Dabei fanden sich sensible Informationen, die das Modell nur indirekt zum Training nutzte.
Durch effizientere Sammelmethoden kann die Anzahl an sensiblen Information sogar noch erhöht werden (Kapitel \ref{sec:data_ext}).


Bei der Poisoning Attacke werden manipulierte Daten in den Datensatz integriert, welche dafür sorgen, dass das Training des Modells sabotiert wird.
Das resultierende Modell wird verschlechtert oder sogar unbrauchbar gemacht.
Werden die manipulierten Daten jedoch bewusst gewählt, kann es sein, dass andere Angriffe wie die Membership Inference Attacke wesentlich effektiver werden (Kapitel \ref{sec:poisoning}).

Verteiltes Lernen bringt einige besondere Herausforderungen mit sich.
Durch das Teilen von beispielsweise Gradienten, können Teilnehmer Rückschlüsse auf Eingabedaten ziehen, was Membership Inference und Property Inference Attacken erleichtert.
Zusätzlich lassen sich mit den Gradienten auch ganze Datensätze rekonstruieren (Kapitel \ref{sec:angriffe_verteiltes_lernen}).