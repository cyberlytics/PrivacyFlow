\section{Abhängigkeit von der Plattform}\label{sec:plattform}

Machine Learning Anwendungen werden oftmals in einer Public Cloud trainiert.
Dafür gibt es eine Vielzahl an Gründen \cite{neumann}:
\begin{compactitem}
    \item \textbf{Rechenleistung}: Die Komplexität eines neuronalen Netzes steigt mit der Komplexität der zu bewältigenden Aufgabe und der Menge an verfügbaren Daten. 
    Mehr Komplexität bedeutet auch, dass mehr Rechenleistung benötigt wird. 
    Eine Public Cloud hat die Möglichkeit, genügend Ressourcen bereitzustellen, um diese Rechenleistung zu bewältigen.
    \item \textbf{Flexibilität}: Das Training eines neuronalen Netzes ist rechenintensiver als die Inferenz von diesem. 
    Durch dir Nutzung einer Public Cloud lässt sich die Rechenleistung so anpassen, dass während dem Betrieb des Modells, weniger Ressourcen genutzt werden, als während des Trainings. 
    Diese Flexibilität sorgt für eine Kosteneffizienz.
    Zusätzlich bieten spezielle Hardware, spezielle Infrastruktur und spezielle Software die Möglichkeit, die Plattform für den Use Case anzupassen.
    \item \textbf{Hardware}: Neuronale Netze werden auf spezieller Hardware trainiert.
    Dazu zählen Grafikkarten, auch Graphics Processing Units oder GPUs genannt, und Tensor Processing Units, auch TPUs genannt.
    Bei der Nutzung der Public Cloud fällt die Anschaffung dieser Hardware weg.
    Stattdessen wird die Nutzungszeit der Hardware berechnet.
    \item \textbf{Integration}: Neben Services für die Entwicklung von Machine Learning Anwendungen gibt es eine Reihe weiterer Services in einer Public Cloud. 
    Diese reichen von Datenbanken bis hin zu Visualisierungen.
    Dabei sind die Services so ausgelegt, dass die Integration verschiedener Services ohne großen Aufwand möglich ist.
\end{compactitem}

Durch die Nutzung einer Public Cloud entstehen jedoch zusätzliche Sicherheitsrisiken.
\mbox{Neben} gängigen Angriffen gegen jede Art von Webanwendungen, wie Ransomware-Angriffe oder Denial-of-Service-Attacken, gibt es eine Reihe weiterer Sicherheitsrisiken, welche die Vertraulichkeit der Daten angreifen \cite{neumann}.

Ein Datenleck ist ein nicht autorisierter Zugriff auf Daten und stellt damit eine große Gefahr bei der Nutzung von sensiblen Daten dar.
Dabei können Daten von einem Angreifer gestohlen werden oder durch andere Fehler an die Öffentlichkeit gelangen.
In einer Public Cloud können diese Datenlecks durch falsche Konfigurationen oder mangelnde Sicherheitsmaßnahmen entstehen \cite{neumann}.
Zusätzlich können Sicherheitslücken in der Software für ein Datenleck sorgen.
Die Sicherheitslücken können dabei in dem eigenen Programmcode integriert sein, oder durch die Nutzung von Software von Drittanbietern stammen \cite{neumann}.

Die Public Cloud ist dabei nur ein Beispiel für die Abhängigkeit von Machine Learning Anwendungen an die darunterliegende Plattform.
Ebenfalls kann der Betrieb eigener Server die gleichen Sicherheitsrisiken aufweisen.
