\section{Anpassung und Betrieb des Modells}\label{sec:betrieb}

Nachdem ein Neuronales Netz trainiert wurde, muss dieses für Nutzer zugänglich gemacht werden.
Häufig wird das Modell dabei auf einem Server bereitgestellt und über eine API direkt angeboten oder in ein bestehendes Produkt integriert.
Dabei ist es möglich, zusätzlich die Vertraulichkeit zu sichern.
Grundsätzlich ist ein Modell, welches über eine API bereitgestellt wird, wie ein Stück Software zu behandeln. 
Best Practices der Softwareentwicklung (wie z. B. Authentifizierung) können und sollten bei Bedarf genutzt werden, um grundlegende Sicherheit zu gewährleisten.
Jedoch gibt es einige Methoden, die speziell für Machine Learning Modelle und damit auch Neuronale Netze genutzt werden können.

Das Ergebnis eines Modells kann beispielsweise mittels Differential Privacy (Kapitel \ref{sec:dp}) verrauscht werden, bevor es über die API zurückgegeben wird. 
Bei einer Regression, wo die Vorhersage ein Zahlenwert ist, könnte entweder der Gauß-Mechanismus oder der Laplace-Mechanimus genutzt werden. 
Bei einer Klassifikation kann der Exponential-Mechanismus genutzt werden.
Die Wahl des Privacy Budget $\epsilon$ ist dabei abhängig von der Sensibilität der Daten.

Im Folgenden werden weitere Methoden betrachtet, welche die Berechnung der Vorhersage verändern.
Dies geschieht beispielsweise durch Nutzung von zusätzlichen kryptografischen Methoden wie Homomorpher Verschlüsselung, aber auch durch eine Transformation des Modells.

\subsection{Kryptografische Inferenz}\label{sec:krypto_inferenz}

Wird ein Modell auf einen fremden Server (beispielsweise in einer Cloud) deployed, wäre es möglich, dass der Provider des Servers Informationen über die Daten, die zur Vorhersage genutzt werden, herausfindet.
Dies wird sowohl durch Homomorphe Verschlüsselung als auch durch Funktionale Verschlüsselung verhindert.
Beide Methoden sorgen dafür, dass Daten verschlüsselt in das Modell gegeben werden, sodass sich diese Daten zu keinem Zeitpunkt unverschlüsselt auf dem fremden Server befinden.
%Ein Unterschied der beiden Methoden liegt jedoch darin, dass bei der Funktionalen Verschlüsselung das Label unverschlüsselt vorliegt, wohingegen bei der Homomorphen Verschlüsselung dieses verschlüsselt existiert.

\subsubsection*{Inferenz mittels Homomorpher Verschlüsselung}

\subsubsection*{Inferenz mittels Funktionaler Verschlüsselung}
% Dufour-Sans et al. \cite{P-105} zeigen, dass es möglich ist, Neuronale Netze mittels Funktionaler Verschlüsselung zu interferieren. 
% Um dies zu ermögliche, stellten die Autoren eine effiziente Funktionale Verschlüsselung auf Basis von Paarungsbasierter Kryptographie vor.

% \subsubsection*{Secure Multi-Party Computation}

% Bei der Secure Multi-Party Computation handelt es sich um einen Forschungsbereich mit dem Ziel, das Teilnehmer gemeinsam eine Funktion berechnen können, ohne dass die einzelnen Eingabewerte aufgedeckt werden. 
% Methoden dieses kryptografischen Forschungsgebiets können auch für Neuronale Netze genutzt werden.

% Rouhani et al. \cite{P-71} stellten ein Framework namens DeepSecure vor, welches Oblivious Transfer, zu Deutsch vergessliche Übertragung, und Garbled Circuits, zu Deutsch verdrehte Schaltkreise, nutzt.
% Oblivious Transfer ist ein kryptografisches Protokoll zwischen einem Sender und einem Empfänger, bei dem der Empfänger einen Index zwischen 1 und $n$ auswählt und der Sender die Nachricht mit dem entsprechenden Index übermittelt. 
% Der Sender weiß dabei jedoch nicht, welcher Index ausgewählt wurde.
% Diese Methodik wir auch 1-aus-$n$ Oblivious Transfer genannt.
% Garbled Circuits ist ebenfalls ein Protokoll, bei der eine Funktion als Boolescher Schaltkreis mit zwei Eingabegattern dargestellt wird.
% Dabei erstellt einer der beiden Teilnehmer, hier Alice genannt, Wahrheitstabellen zu jedem Logikgatter des Schaltkreises. 
% Die Inputs sind dabei nicht 0 und 1, sondern jeweils eine Folge von $k$ randomisierten Bits, welche 0 und 1 kodieren.
% Die Ergebnisspalte dieser Wahrheitstabellen verschlüsselt Alice anschließend mit den beiden Inputs, sodass dies nur mit den beiden Inputs wieder entschlüsselt werden kann. 
% Zusätzlich wird die Reihenfolge der Zeilen randomisiert, damit aufgrund der Reihenfolge keine Rückschlüsse gewonnen werden können. 
% Dieser Schritt wird Garbling genannt und die entstandenen Tabellen sind sogenannte Garbled Tabellen.
% Anschließend überträgt Alice die Garbled Tabellen an den zweiten Teilnehmer, hier Bob.
% Mittels 1-aus-2 Oblivious Transfer wählt Bob eine von zwei Nachrichten aus, wobei der Index seinem Input entspricht und die zwei Nachrichten die kodierten Labels von Alice sind.
% Die erhaltene Nachricht und das eigene Label können nun genutzt werden, um die Ergebnisspalte einer Garbled Tabelle zu entschlüsseln.
% Bob führt dies für jedes Gatter des Schaltkreises aus.
% Am Ende erhält Bob den Output des letzten Gatters, welchen jedoch einer der randomisierten Bitfolgen ist. 
% Er übermittelt diesen an Alice und erhält dadurch den entsprechenden 0 oder 1 Wert.

% DeepSecure 
\subsection{Kompression des Modells}\label{sec:kompression}

Eigentlich dient die Kompression eines Modells dazu, den Speicherverbrauch zu minimieren und zusätzlich Rechenleistung bei der Vorhersage zu sparen.
Jedoch gibt es auch einige Ansätze, wie Modellkompression genutzt werden kann, um die Vertraulichkeit der Daten zu sichern.

Ein Ansatz der Modellkompression ist es, ein Teacher-Modell zu trainieren und dieses dann dazu zu nutzen, ein Student-Modell zu trainieren. 
Die in Kapitel \ref{sec:pate} beschriebene Methode PATE nutzt ebenfalls  eine Teacher-Student-Architektur. 
Jedoch erfordert PATE eine Anpassung des Trainingsprozesses, indem verschiedene Teacher Modelle trainiert werden.
Andere Techniken können ein bestehendes Modell als Teacher nutzen.

Die Destillation eines Modells wurde erstmals von Hinton et al. \cite{P-61} vorgestellt.
Dabei handelt es sich auch um eine Teacher-Student-Architektur, bei welcher ein einzelnes Modell, wie auch ein Ensemble an Modellen als Teacher genutzt werden kann.
Das Student-Modell, welches eine ähnliche Architektur wie das Teacher-Modell hat, soll dabei lernen, die gleiche Wahrscheinlichkeitsverteilung wie das Teacher-Modell vorherzusagen.
Anschließend wird nur das Student-Modell genutzt, um Vorhersagen zu berechnen.
Für das Training des Studen- Modells kann der gleiche Trainingsdatenbestand genutzt werden, jedoch ist auch ein alternativer Datenbestand möglich.
Als Label werden die Vorhersagen des Teacher-Modells, beziehungsweise die aggregierte Vorhersage des Teacher-Ensembles.
Klassifikatoren haben in der Regel eine Softmax Aktivierungsfunktion in der letzten Schicht, welche die Wahrscheinlichkeiten der einzelnen Klassen ausgibt.
Die Softmax Funktion hat dabei einen Parameter namens Temperatur, welcher die Entropie der Wahrscheinlichkeiten beeinflusst. 
Normalerweise ist der Wert der Temperatur auf 1 gesetzt, was dafür sorgt, dass die Wahrscheinlichkeit der vorhergesagten Klasse deutlich größer als die anderen Wahrscheinlichkeiten ist.
Eine höhere Temperatur hat zur Folge, dass sich die Wahrscheinlichkeiten annähern und die Verteilung dadurch glatter wird.
Modell Destillation nutzt eine höhere Temperatur im Teacher-Modell zum Labeln der Datensätze und die gleiche Temperatur während des Trainings des Student-Modells.
Dies sorgt dafür, dass das Student-Modell die Verteilungen besser lernen kann, da so auch nicht vorhergesagte Klassen mehr Einfluss auf die Gradienten haben.
Nach dem Training nutzt das Student-Modell wieder eine Temperatur von 1.
Die Autoren zeigen, dass die Temperatur einen deutlichen Einfluss auf die Güte des Modells haben kann. 
Der Wert kann dabei zwischen 2,5 und 20 schwanken.
Wang et al. \cite{P-64} zeigen, dass Modell Destillation in Kombination mit Differential Privacy genutzt werden kann, um ein Student-Modell zu erhalten, welches die Vertraulichkeit der Daten schützt. 
Dabei werden die Outputs der Softmax Funktion mit hoher Temperatur des Teacher Modells mit dem Gauß-Mechanismus verrauscht, bevor diese als Label für das Student Modell genutzt werden.

Die Methode der Destillation wurde von Polino et al. \cite{P-62} durch die sogenannte Quantisierung erweitert.
Ziel von Quantisierung ist, die Gewichte des Modells mit in einer festgelegten Bit-Länge anzugeben.
Dabei werden die möglichen, kontinuierlichen Werte in den Wertebereich $[0,1]$ projiziert und können anschließend in einen Zielwertebereich (mit festgelegter Bit-Länge) skaliert werden.
Die Skalierung erfolgt dabei anhand einer Gleichverteilung.
Dafür werden die kontinuierlichen Werte im Wertebereich $[0,1]$ in Quantisierungsintervalle eingeteilt, wobei die Anzahl der Intervalle gleich der Anzahl an Bits im Zielbereich ist.
Jeder Wert wird dem nähesten dieser Intervalle zugeordnet.
Es ist dabei anzumerken, dass durch diese Intervalleinteilung ein Rundungsfehler entsteht.
Dieser Fehler entspricht dem Rauschen einer Gauß Verteilung.
Dies ähnelt dem Rauschen des Gauß-Mechanismus von Differential Privacy und könnte die Vertraulichkeit schützen.
Die Autoren gehen aber nicht weiter auf das Thema ein und es gibt auch keine Berechnung eines Privacy Budgets.
Quantisierung kann genutzt werden, um die Größe eines Modells zu reduzieren.
Die Autoren zeigen, dass Quantisierung auch in Kombination mit Modell Destillation genutzt werden kann, wodurch das Student Modell noch kleiner im Vergleich zum ursprünglichen Teacher-Modell wird, obwohl die Genauigkeit nahezu gleich bleibt.


