\subsection{Homomorphe Verschlüsselung}

Homomorphe Verschlüsselung ermöglicht die Berechnung auf verschlüsselten Daten. Dadurch können Daten beispielsweise in der Cloud verarbeitet werden, ohne dass es dem Anbieter des Services möglich ist, die Vertraulichkeit der Daten zu gefährden.

Ein Homomorphismus in der Algebra bezeichnet eine strukturerhaltende Abbildung einer Mengen $G$ in eine andere Menge $H$.
Dabei hat jedes Element $g \in G $ mindestens ein Bild $h \in H$ und die Relationen der Elemente $g \in G$ zueinander, finden sich auch in H wieder \cite{B-2}.

Ein typisches Beispiel ist der Homomorphismus zwischen zwei Gruppen $(G,\circ)$ und $(H,\ast)$. 
Die Beziehung der beiden Gruppen wird Gruppenhomomorphismus genannt, wenn es eine Funktion $f:G\to H$ gibt, die Elemente der Gruppe $G$ auf die Gruppe $H$ abbildet und dabei für alle Elemente $g_1,g_2 \in G$ gilt \cite{P-98}:
\begin{equation*}
    f(g_1 \circ g_2) = f(g_1) \ast f(g_2)
\end{equation*}
Abbildung \ref{fig:group_homomorphismus} zeigt eine grafische Darstellung dieses Gruppenhomomorphismus.

\begin{figure}[!htb]
    \centering
    \includegraphics[width=12cm]{figures/group_homomophismus.png}
    \caption{Gruppenhomomorphismus nach \cite{P-98}}
    \label{fig:group_homomorphismus}
\end{figure} 

Bei der homomorphen Verschlüsselung, handelt es sich um einen Gruppenhomomorphismus zwischen der Gruppe der Klartexte $(P,\circ)$ und der Gruppe der Geheimtexte $(C,\ast)$. 
Die Abbildfunktionen sind dabei der Verschlüsselungsalgorithmus $E_k:P\to C$ und der Entschlüsselungsalgorithmus $D_k:C\to P$ mit einem Schlüssel $k \in K$ \cite{P-98}. 
Daraus lässt sich ableiten, dass folgende Bedingungen erfüllt sind:
\begin{equation*}
    E_k(p_1 \circ p_2) = E_k(p_1) \ast E_k(p_2)  \text{ und } D_k(c_1 \ast c_2) = D_k(c_1) \circ D_k(c_2)
\end{equation*}
Abbildung \ref{fig:homo_enc} zeigt, wie besagter Homomorphismus aussieht.

\begin{figure}[!htb]
    \centering
    \includegraphics[width=12cm]{figures/homo_enc.png}
    \caption{Homomorphe Verschlüsselung}
    \label{fig:homo_enc}
\end{figure} 

Homomorphe Verschlüsselungen lassen sich dabei in 3 Kategorien einteilen, je nachdem welche Verknüpfungen innerhalb der Gruppen möglich sind \cite{P-42}:
\begin{compactitem}
\item \textbf{Teilweise homomorphe Verschlüsselung (partially):} Entweder Multiplikation oder Addition möglich, jedoch nicht beides.
\item \textbf{Eingeschränkte homomorphe Verschlüsselung (somewhat):} Sowohl Multiplikation als auch Addition möglich, jedoch beschränkt durch die Anzahl an durchführbaren Berechnungen. 
\item \textbf{Vollständige homomorphe Verschlüsselung (fully):} Multiplikation und Addition für eine unbegrenzte Anzahl an Berechnungen möglich
\end{compactitem}

Gentry \cite{P-40} stellte 2009 das erste vollständig homomorphe Verschlüsselungssystem vor.
Dabei nutzte er eine eingeschränkt homomorphe Verschlüsselung, welche auf mathematischen Gittern basiert.
Das System war eingeschränkt homomorph, da die Verschlüsselung auf einem Rauschen basierte, welches mit jeder Operation größer wurde und letztendlich nicht mehr für eine korrekte Entschlüsselung sorgte.
Er erweiterte das System mit einer Technik namens Bootstrapping.
Dabei wird der Geheimtext ein zweites Mal verschlüsselt, sodass dieser doppelt verschlüsselt ist.
Anschließend kann mittels des verschlüsselten Schlüssels die ursprüngliche Verschlüsselung homomorph herausgerechnet werden. 
Dadurch wird das Rauschen den Verschlüsselungssystems zurückgesetzt und eine weitere Berechnung ist möglich. 
Kann das ursprünglich eingeschränkte homomorphe Verschlüsselungssystem die homomorphe Entschlüsselung und eine weitere Operation durchführen, dann kann es mittels Bootstrapping zu einem vollständig homomorphen System umgewandelt werden.
Van Dijk et al. \cite{P-100} tauschten die auf Gittern basierte Verschlüsselung durch eine auf Ganzzahlen basierte, eingeschränkt homomorphe Verschlüsselung aus.



