\chapter{Ergebnisse im Detail}\label{ch:ergebnisse_detail}
Im Folgenden befinden sich die Messungen der Experimente in ausführlicher Fassung.
Einige Werte wurden bereits in Kapitel \ref{ch:experiments} beschrieben.

\section{Hyperparametertuning DPSGD CIFAR-10 Modell}
Tabelle \ref{tab:c10_base} zeigt die Genauigkeit des CIFAR-10 Modells ohne die Nutzung von Differential Privacy.
Dabei werden verschiedene Batch-Größen und Anzahl an trainierten Epochen betrachtet.
Zusätzlich ist Datenaugmentierung mittels AutoAugment aktiviert, welche in \ref{tab:c10_base2} nicht genutzt wird.
In Tabelle \ref{tab:c10_exp1} wird DPSGD benutzt, jedoch ohne eine Anpassung von Parametern. 
Die Clipping-Norm ist auf 1,0 gesetzt und Datenaugmentierung wird in Form von AutoAugment genutzt.
Die Datenaugmentierung fällt in Tabelle \ref{tab:c10_exp2} jedoch weg.
Tabelle \ref{tab:c10_exp3} zeigt die Genauigkeit des Modells mit einer angepassten Clipping-Norm von $10^{-5}$.
\begin{table}[!htb]
\centering
\begin{tabular}{|l|l|l|l|l|}
\hline
\rowcolor[HTML]{CBCEFB} 
Epsilon  & \begin{tabular}[c]{@{}l@{}}Anzahl\\ Epochen\end{tabular} & \begin{tabular}[c]{@{}l@{}}Genauigkeit mit\\ Batch-Größe=64\end{tabular} & \begin{tabular}[c]{@{}l@{}}Genauigkeit mit\\ Batch-Größe=128\end{tabular} & \begin{tabular}[c]{@{}l@{}}Genauigkeit mit\\ Batch-Größe=256\end{tabular} \\ \hline
$\infty$ & 10 & 81,3 \% & 82,6 \% & 79,8 \% \\ \hline
$\infty$ & 12 & 82,4 \% & 82,8 \% & 81,8 \% \\ \hline
$\infty$ & 15 & 84,6 \% & 82,5 \% & 83,1 \% \\ \hline
$\infty$ & 20 & 84,5 \% & 84,1 \% & 82,5 \% \\ \hline
\end{tabular}
\caption{CIFAR-10 Modell mit AutoAugment ohne Differential Privacy }
\label{tab:c10_base}
\end{table}
\begin{table}[!htb]
\centering
\begin{tabular}{|l|l|l|l|l|}
\hline
\rowcolor[HTML]{CBCEFB} 
Epsilon  & \begin{tabular}[c]{@{}l@{}}Anzahl\\ Epochen\end{tabular} & \begin{tabular}[c]{@{}l@{}}Genauigkeit mit\\ Batch-Größe=64\end{tabular} & \begin{tabular}[c]{@{}l@{}}Genauigkeit mit\\ Batch-Größe=128\end{tabular} & \begin{tabular}[c]{@{}l@{}}Genauigkeit mit\\ Batch-Größe=256\end{tabular} \\ \hline
$\infty$ & 10 & 82,2 \% & 78,3 \% & 75,7 \% \\ \hline
$\infty$ & 12 & 79,6 \% & 81,9 \% & 80,6 \% \\ \hline
$\infty$ & 15 & 78,8 \% & 80,2 \% & 80,1 \% \\ \hline
$\infty$ & 20 & 79,5 \% & 82,9 \% & 80,9 \% \\ \hline  
\end{tabular}
\caption{CIFAR-10 Modell ohne Augmentierung ohne Differential Privacy }
\label{tab:c10_base2}
\end{table}
\begin{table}[!htb]
\centering
\begin{tabular}{lllll}
\hline
\rowcolor[HTML]{EFEFEF} 
\multicolumn{5}{|l|}{\cellcolor[HTML]{EFEFEF}CIFAR-10 Modell mit Delta=$10^{-5}$, mit AutoAugments und Clipping bei 1,0}                                                                                                                                                                                                                                                                                                                                                                                                              \\ \hline
\rowcolor[HTML]{CBCEFB} 
\multicolumn{1}{|l|}{\cellcolor[HTML]{CBCEFB}{\color[HTML]{000000} Epsilon}} & \multicolumn{1}{l|}{\cellcolor[HTML]{CBCEFB}{\color[HTML]{000000} \begin{tabular}[c]{@{}l@{}}Anzahl \\ Epochen\end{tabular}}} & \multicolumn{1}{l|}{\cellcolor[HTML]{ECF4FF}\begin{tabular}[c]{@{}l@{}}Genauigkeit mit\\ Batch-Größe=64\end{tabular}} & \multicolumn{1}{l|}{\cellcolor[HTML]{ECF4FF}\begin{tabular}[c]{@{}l@{}}Genauigkeit mit\\ Batch-Größe=128\end{tabular}} & \multicolumn{1}{l|}{\cellcolor[HTML]{ECF4FF}\begin{tabular}[c]{@{}l@{}}Genauigkeit mit\\ Batch-Größe=256\end{tabular}} \\ \hline
\multicolumn{1}{|l|}{1}                                                      & \multicolumn{1}{l|}{1}                                                                 & \multicolumn{1}{l|}{13,9 \%}                                                                                           & \multicolumn{1}{l|}{16,6 \%}                                                                                            & \multicolumn{1}{l|}{24,7 \%}                                                                                            \\ \hline
\multicolumn{1}{|l|}{1}                                                      & \multicolumn{1}{l|}{5}                                                                 & \multicolumn{1}{l|}{23,3 \%}                                                                                           & \multicolumn{1}{l|}{28,1 \%}                                                                                            & \multicolumn{1}{l|}{36,0 \%}                                                                                            \\ \hline
\multicolumn{1}{|l|}{1}                                                      & \multicolumn{1}{l|}{10}                                                                & \multicolumn{1}{l|}{18,1 \%}                                                                                           & \multicolumn{1}{l|}{29,8 \%}                                                                                            & \multicolumn{1}{l|}{34,9 \%}                                                                                            \\ \hline
\multicolumn{1}{|l|}{1}                                                      & \multicolumn{1}{l|}{15}                                                                & \multicolumn{1}{l|}{24,0 \%}                                                                                           & \multicolumn{1}{l|}{31,2 \%}                                                                                            & \multicolumn{1}{l|}{36,6 \%}                                                                                            \\ \hline
\multicolumn{1}{|l|}{1}                                                      & \multicolumn{1}{l|}{20}                                                                & \multicolumn{1}{l|}{18,4 \%}                                                                                           & \multicolumn{1}{l|}{28,4 \%}                                                                                            & \multicolumn{1}{l|}{34,9 \%}                                                                                            \\ \hline
\multicolumn{1}{|l|}{1}                                                      & \multicolumn{1}{l|}{30}                                                                & \multicolumn{1}{l|}{17,6 \%}                                                                                           & \multicolumn{1}{l|}{30,9 \%}                                                                                            & \multicolumn{1}{l|}{36,9 \%}                                                                                            \\ \hline
                                                                             &                                                                                        &                                                                                                                       &                                                                                                                        &                                                                                                                        \\ \hline
\multicolumn{1}{|l|}{5}                                                      & \multicolumn{1}{l|}{1}                                                                 & \multicolumn{1}{l|}{18,8 \%}                                                                                           & \multicolumn{1}{l|}{28,7 \%}                                                                                            & \multicolumn{1}{l|}{27,1 \%}                                                                                            \\ \hline
\multicolumn{1}{|l|}{5}                                                      & \multicolumn{1}{l|}{5}                                                                 & \multicolumn{1}{l|}{30,6 \%}                                                                                           & \multicolumn{1}{l|}{38,9 \%}                                                                                            & \multicolumn{1}{l|}{41,7 \%}                                                                                            \\ \hline
\multicolumn{1}{|l|}{5}                                                      & \multicolumn{1}{l|}{10}                                                                & \multicolumn{1}{l|}{26,1 \%}                                                                                           & \multicolumn{1}{l|}{40,1 \%}                                                                                            & \multicolumn{1}{l|}{43,3 \%}                                                                                            \\ \hline
\multicolumn{1}{|l|}{5}                                                      & \multicolumn{1}{l|}{15}                                                                & \multicolumn{1}{l|}{30,7 \%}                                                                                           & \multicolumn{1}{l|}{40,7 \%}                                                                                            & \multicolumn{1}{l|}{47,0 \%}                                                                                            \\ \hline
\multicolumn{1}{|l|}{5}                                                      & \multicolumn{1}{l|}{20}                                                                & \multicolumn{1}{l|}{29,2 \%}                                                                                           & \multicolumn{1}{l|}{39,3 \%}                                                                                            & \multicolumn{1}{l|}{45,4 \%}                                                                                            \\ \hline
\multicolumn{1}{|l|}{5}                                                      & \multicolumn{1}{l|}{30}                                                                & \multicolumn{1}{l|}{32,7 \%}                                                                                           & \multicolumn{1}{l|}{41,4 \%}                                                                                            & \multicolumn{1}{l|}{47,4 \%}                                                                                            \\ \hline
                                                                             &                                                                                        &                                                                                                                       &                                                                                                                        &                                                                                                                        \\ \hline
\multicolumn{1}{|l|}{10}                                                     & \multicolumn{1}{l|}{1}                                                                 & \multicolumn{1}{l|}{17,8 \%}                                                                                           & \multicolumn{1}{l|}{26,3 \%}                                                                                            & \multicolumn{1}{l|}{30,6 \%}                                                                                            \\ \hline
\multicolumn{1}{|l|}{10}                                                     & \multicolumn{1}{l|}{5}                                                                 & \multicolumn{1}{l|}{31,7 \%}                                                                                           & \multicolumn{1}{l|}{36,7 \%}                                                                                            & \multicolumn{1}{l|}{42,9 \%}                                                                                            \\ \hline
\multicolumn{1}{|l|}{10}                                                     & \multicolumn{1}{l|}{10}                                                                & \multicolumn{1}{l|}{34,4 \%}                                                                                           & \multicolumn{1}{l|}{39,7 \%}                                                                                            & \multicolumn{1}{l|}{46,6 \%}                                                                                            \\ \hline
\multicolumn{1}{|l|}{10}                                                     & \multicolumn{1}{l|}{15}                                                                & \multicolumn{1}{l|}{32,2 \%}                                                                                           & \multicolumn{1}{l|}{41,5 \%}                                                                                            & \multicolumn{1}{l|}{48,8 \%}                                                                                            \\ \hline
\multicolumn{1}{|l|}{10}                                                     & \multicolumn{1}{l|}{20}                                                                & \multicolumn{1}{l|}{29,1 \%}                                                                                           & \multicolumn{1}{l|}{42,0 \%}                                                                                            & \multicolumn{1}{l|}{50,6 \%}                                                                                            \\ \hline
\multicolumn{1}{|l|}{10}                                                     & \multicolumn{1}{l|}{30}                                                                & \multicolumn{1}{l|}{32,5 \%}                                                                                           & \multicolumn{1}{l|}{43,1 \%}                                                                                            & \multicolumn{1}{l|}{52,9 \%}                                                                                            \\ \hline
                                                                             &                                                                                        &                                                                                                                       &                                                                                                                        &                                                                                                                        \\ \hline
\multicolumn{1}{|l|}{20}                                                     & \multicolumn{1}{l|}{1}                                                                 & \multicolumn{1}{l|}{27,6 \%}                                                                                           & \multicolumn{1}{l|}{31,3 \%}                                                                                            & \multicolumn{1}{l|}{19,4 \%}                                                                                            \\ \hline
\multicolumn{1}{|l|}{20}                                                     & \multicolumn{1}{l|}{5}                                                                 & \multicolumn{1}{l|}{34,6 \%}                                                                                           & \multicolumn{1}{l|}{39,3 \%}                                                                                            & \multicolumn{1}{l|}{43,1 \%}                                                                                            \\ \hline
\multicolumn{1}{|l|}{20}                                                     & \multicolumn{1}{l|}{10}                                                                & \multicolumn{1}{l|}{33,9 \%}                                                                                           & \multicolumn{1}{l|}{44,5 \%}                                                                                            & \multicolumn{1}{l|}{48,1 \%}                                                                                            \\ \hline
\multicolumn{1}{|l|}{20}                                                     & \multicolumn{1}{l|}{15}                                                                & \multicolumn{1}{l|}{33,1 \%}                                                                                           & \multicolumn{1}{l|}{43,9 \%}                                                                                            & \multicolumn{1}{l|}{52,6 \%}                                                                                            \\ \hline
\multicolumn{1}{|l|}{20}                                                     & \multicolumn{1}{l|}{20}                                                                & \multicolumn{1}{l|}{36,2 \%}                                                                                           & \multicolumn{1}{l|}{45,9 \%}                                                                                            & \multicolumn{1}{l|}{51,4 \%}                                                                                            \\ \hline
\multicolumn{1}{|l|}{20}                                                     & \multicolumn{1}{l|}{30}                                                                & \multicolumn{1}{l|}{37,2 \%}                                                                                           & \multicolumn{1}{l|}{48,9 \%}                                                                                            & \multicolumn{1}{l|}{55,7 \%}                                                                                            \\ \hline
                                                                             &                                                                                        &                                                                                                                       &                                                                                                                        &                                                                                                                        \\ \hline
\multicolumn{1}{|l|}{30}                                                     & \multicolumn{1}{l|}{1}                                                                 & \multicolumn{1}{l|}{27,5 \%}                                                                                           & \multicolumn{1}{l|}{29,2 \%}                                                                                            & \multicolumn{1}{l|}{29,3 \%}                                                                                            \\ \hline
\multicolumn{1}{|l|}{30}                                                     & \multicolumn{1}{l|}{5}                                                                 & \multicolumn{1}{l|}{37,7 \%}                                                                                           & \multicolumn{1}{l|}{40,6 \%}                                                                                            & \multicolumn{1}{l|}{46,0 \%}                                                                                            \\ \hline
\multicolumn{1}{|l|}{30}                                                     & \multicolumn{1}{l|}{10}                                                                & \multicolumn{1}{l|}{36,7 \%}                                                                                           & \multicolumn{1}{l|}{42,7 \%}                                                                                            & \multicolumn{1}{l|}{50,8 \%}                                                                                            \\ \hline
\multicolumn{1}{|l|}{30}                                                     & \multicolumn{1}{l|}{15}                                                                & \multicolumn{1}{l|}{38,1 \%}                                                                                           & \multicolumn{1}{l|}{50,2 \%}                                                                                            & \multicolumn{1}{l|}{53,7 \%}                                                                                            \\ \hline
\multicolumn{1}{|l|}{30}                                                     & \multicolumn{1}{l|}{20}                                                                & \multicolumn{1}{l|}{36,1 \%}                                                                                           & \multicolumn{1}{l|}{49,2 \%}                                                                                            & \multicolumn{1}{l|}{55,5 \%}                                                                                            \\ \hline
\multicolumn{1}{|l|}{30}                                                     & \multicolumn{1}{l|}{30}                                                                & \multicolumn{1}{l|}{40,2 \%}                                                                                           & \multicolumn{1}{l|}{50,0 \%}                                                                                            & \multicolumn{1}{l|}{56,9 \%}                                                                                            \\ \hline
                                                                             &                                                                                        &                                                                                                                       &                                                                                                                        &                                                                                                                        \\ \hline
\multicolumn{1}{|l|}{50}                                                     & \multicolumn{1}{l|}{1}                                                                 & \multicolumn{1}{l|}{33,0 \%}                                                                                           & \multicolumn{1}{l|}{30,2 \%}                                                                                            & \multicolumn{1}{l|}{23,2 \%}                                                                                            \\ \hline
\multicolumn{1}{|l|}{50}                                                     & \multicolumn{1}{l|}{5}                                                                 & \multicolumn{1}{l|}{40,4 \%}                                                                                           & \multicolumn{1}{l|}{43,6 \%}                                                                                            & \multicolumn{1}{l|}{48,1 \%}                                                                                            \\ \hline
\multicolumn{1}{|l|}{50}                                                     & \multicolumn{1}{l|}{10}                                                                & \multicolumn{1}{l|}{38,7 \%}                                                                                           & \multicolumn{1}{l|}{48,8 \%}                                                                                            & \multicolumn{1}{l|}{53,3 \%}                                                                                            \\ \hline
\multicolumn{1}{|l|}{50}                                                     & \multicolumn{1}{l|}{15}                                                                & \multicolumn{1}{l|}{38,9 \%}                                                                                           & \multicolumn{1}{l|}{52,4 \%}                                                                                            & \multicolumn{1}{l|}{54,0 \%}                                                                                            \\ \hline
\multicolumn{1}{|l|}{50}                                                     & \multicolumn{1}{l|}{20}                                                                & \multicolumn{1}{l|}{42,7 \%}                                                                                           & \multicolumn{1}{l|}{53,8 \%}                                                                                            & \multicolumn{1}{l|}{56,6 \%}                                                                                            \\ \hline
\multicolumn{1}{|l|}{50}                                                     & \multicolumn{1}{l|}{30}                                                                & \multicolumn{1}{l|}{45,1 \%}                                                                                           & \multicolumn{1}{l|}{52,3 \%}                                                                                            & \multicolumn{1}{l|}{59,1 \%}                                                                                            \\ \hline
\end{tabular}
\caption{CIFAR-10 Modell, DPSGD mit Clipping bei 1,0 und mit AutoAugments}
\label{tab:c10_exp1}
\end{table}
\begin{table}[!htb]
\centering
\begin{tabular}{lllll}
\hline
\rowcolor[HTML]{EFEFEF} 
\multicolumn{5}{|l|}{\cellcolor[HTML]{EFEFEF}CIFAR-10 Modell mit Delta=$10^{-5}$ und Clipping bei 1,0}                                                                                                                                                                                                                                                                                                                                                                                                                                                                   \\ \hline
\rowcolor[HTML]{CBCEFB} 
\multicolumn{1}{|l|}{\cellcolor[HTML]{CBCEFB}{\color[HTML]{000000} Epsilon}} & \multicolumn{1}{l|}{\cellcolor[HTML]{CBCEFB}{\color[HTML]{000000} \begin{tabular}[c]{@{}l@{}}Anzahl \\ Epochen\end{tabular}}} & \multicolumn{1}{l|}{\cellcolor[HTML]{CBCEFB}\begin{tabular}[c]{@{}l@{}}Genauigkeit mit\\ Batch-Größe=64\end{tabular}} & \multicolumn{1}{l|}{\cellcolor[HTML]{CBCEFB}\begin{tabular}[c]{@{}l@{}}Genauigkeit mit\\ Batch-Größe=128\end{tabular}} & \multicolumn{1}{l|}{\cellcolor[HTML]{CBCEFB}\begin{tabular}[c]{@{}l@{}}Genauigkeit mit\\ Batch-Größe=256\end{tabular}} \\ \hline
\multicolumn{1}{|l|}{1}                                                      & \multicolumn{1}{l|}{1}                                                                                                        & \multicolumn{1}{l|}{24,7 \%}                                                                                          & \multicolumn{1}{l|}{27,7 \%}                                                                                           & \multicolumn{1}{l|}{32,6 \%}                                                                                           \\ \hline
\multicolumn{1}{|l|}{1}                                                      & \multicolumn{1}{l|}{5}                                                                                                        & \multicolumn{1}{l|}{30,2 \%}                                                                                          & \multicolumn{1}{l|}{35,9 \%}                                                                                           & \multicolumn{1}{l|}{43,4 \%}                                                                                           \\ \hline
\multicolumn{1}{|l|}{1}                                                      & \multicolumn{1}{l|}{10}                                                                                                       & \multicolumn{1}{l|}{35,2 \%}                                                                                          & \multicolumn{1}{l|}{40,7 \%}                                                                                           & \multicolumn{1}{l|}{44,9 \%}                                                                                           \\ \hline
\multicolumn{1}{|l|}{1}                                                      & \multicolumn{1}{l|}{15}                                                                                                       & \multicolumn{1}{l|}{34,3 \%}                                                                                          & \multicolumn{1}{l|}{41,0 \%}                                                                                           & \multicolumn{1}{l|}{44,6 \%}                                                                                           \\ \hline
\multicolumn{1}{|l|}{1}                                                      & \multicolumn{1}{l|}{20}                                                                                                       & \multicolumn{1}{l|}{32,6 \%}                                                                                          & \multicolumn{1}{l|}{41,0 \%}                                                                                           & \multicolumn{1}{l|}{43,9 \%}                                                                                           \\ \hline
\multicolumn{1}{|l|}{1}                                                      & \multicolumn{1}{l|}{30}                                                                                                       & \multicolumn{1}{l|}{26,5 \%}                                                                                          & \multicolumn{1}{l|}{39,5 \%}                                                                                           & \multicolumn{1}{l|}{44,5 \%}                                                                                           \\ \hline
                                                                             &                                                                                                                               &                                                                                                                       &                                                                                                                        &                                                                                                                        \\ \hline
\multicolumn{1}{|l|}{5}                                                      & \multicolumn{1}{l|}{1}                                                                                                        & \multicolumn{1}{l|}{34,5 \%}                                                                                          & \multicolumn{1}{l|}{30,6 \%}                                                                                           & \multicolumn{1}{l|}{34,5 \%}                                                                                           \\ \hline
\multicolumn{1}{|l|}{5}                                                      & \multicolumn{1}{l|}{5}                                                                                                        & \multicolumn{1}{l|}{38,8 \%}                                                                                          & \multicolumn{1}{l|}{40,9 \%}                                                                                           & \multicolumn{1}{l|}{47,5 \%}                                                                                           \\ \hline
\multicolumn{1}{|l|}{5}                                                      & \multicolumn{1}{l|}{10}                                                                                                       & \multicolumn{1}{l|}{40,0 \%}                                                                                          & \multicolumn{1}{l|}{44,3 \%}                                                                                           & \multicolumn{1}{l|}{48,7 \%}                                                                                           \\ \hline
\multicolumn{1}{|l|}{5}                                                      & \multicolumn{1}{l|}{15}                                                                                                       & \multicolumn{1}{l|}{40,9 \%}                                                                                          & \multicolumn{1}{l|}{47,1 \%}                                                                                           & \multicolumn{1}{l|}{50,6 \%}                                                                                           \\ \hline
\multicolumn{1}{|l|}{5}                                                      & \multicolumn{1}{l|}{20}                                                                                                       & \multicolumn{1}{l|}{40,5 \%}                                                                                          & \multicolumn{1}{l|}{46,1 \%}                                                                                           & \multicolumn{1}{l|}{51,9 \%}                                                                                           \\ \hline
\multicolumn{1}{|l|}{5}                                                      & \multicolumn{1}{l|}{30}                                                                                                       & \multicolumn{1}{l|}{40,5 \%}                                                                                          & \multicolumn{1}{l|}{44,9 \%}                                                                                           & \multicolumn{1}{l|}{52,0 \%}                                                                                           \\ \hline
                                                                             &                                                                                                                               &                                                                                                                       &                                                                                                                        &                                                                                                                        \\ \hline
\multicolumn{1}{|l|}{10}                                                     & \multicolumn{1}{l|}{1}                                                                                                        & \multicolumn{1}{l|}{32,6 \%}                                                                                          & \multicolumn{1}{l|}{38,5 \%}                                                                                           & \multicolumn{1}{l|}{35,7 \%}                                                                                           \\ \hline
\multicolumn{1}{|l|}{10}                                                     & \multicolumn{1}{l|}{5}                                                                                                        & \multicolumn{1}{l|}{39,6 \%}                                                                                          & \multicolumn{1}{l|}{43,3 \%}                                                                                           & \multicolumn{1}{l|}{48,6 \%}                                                                                           \\ \hline
\multicolumn{1}{|l|}{10}                                                     & \multicolumn{1}{l|}{10}                                                                                                       & \multicolumn{1}{l|}{39,7 \%}                                                                                          & \multicolumn{1}{l|}{47,1 \%}                                                                                           & \multicolumn{1}{l|}{50,6 \%}                                                                                           \\ \hline
\multicolumn{1}{|l|}{10}                                                     & \multicolumn{1}{l|}{15}                                                                                                       & \multicolumn{1}{l|}{40,4 \%}                                                                                          & \multicolumn{1}{l|}{47,1 \%}                                                                                           & \multicolumn{1}{l|}{53,3 \%}                                                                                           \\ \hline
\multicolumn{1}{|l|}{10}                                                     & \multicolumn{1}{l|}{20}                                                                                                       & \multicolumn{1}{l|}{39,9 \%}                                                                                          & \multicolumn{1}{l|}{49,8 \%}                                                                                           & \multicolumn{1}{l|}{54,0 \%}                                                                                           \\ \hline
\multicolumn{1}{|l|}{10}                                                     & \multicolumn{1}{l|}{30}                                                                                                       & \multicolumn{1}{l|}{41,7 \%}                                                                                          & \multicolumn{1}{l|}{50,8 \%}                                                                                           & \multicolumn{1}{l|}{54,4 \%}                                                                                           \\ \hline
                                                                             &                                                                                                                               &                                                                                                                       &                                                                                                                        &                                                                                                                        \\ \hline
\multicolumn{1}{|l|}{20}                                                     & \multicolumn{1}{l|}{1}                                                                                                        & \multicolumn{1}{l|}{34,3 \%}                                                                                          & \multicolumn{1}{l|}{37,6 \%}                                                                                           & \multicolumn{1}{l|}{37,5 \%}                                                                                           \\ \hline
\multicolumn{1}{|l|}{20}                                                     & \multicolumn{1}{l|}{5}                                                                                                        & \multicolumn{1}{l|}{44,0 \%}                                                                                          & \multicolumn{1}{l|}{47,3 \%}                                                                                           & \multicolumn{1}{l|}{48,7 \%}                                                                                           \\ \hline
\multicolumn{1}{|l|}{20}                                                     & \multicolumn{1}{l|}{10}                                                                                                       & \multicolumn{1}{l|}{45,4 \%}                                                                                          & \multicolumn{1}{l|}{48,5 \%}                                                                                           & \multicolumn{1}{l|}{53,2 \%}                                                                                           \\ \hline
\multicolumn{1}{|l|}{20}                                                     & \multicolumn{1}{l|}{15}                                                                                                       & \multicolumn{1}{l|}{45,8 \%}                                                                                          & \multicolumn{1}{l|}{51,1 \%}                                                                                           & \multicolumn{1}{l|}{55,0 \%}                                                                                           \\ \hline
\multicolumn{1}{|l|}{20}                                                     & \multicolumn{1}{l|}{20}                                                                                                       & \multicolumn{1}{l|}{45,6 \%}                                                                                          & \multicolumn{1}{l|}{52,4 \%}                                                                                           & \multicolumn{1}{l|}{56,1 \%}                                                                                           \\ \hline
\multicolumn{1}{|l|}{20}                                                     & \multicolumn{1}{l|}{30}                                                                                                       & \multicolumn{1}{l|}{45,2 \%}                                                                                          & \multicolumn{1}{l|}{54,0 \%}                                                                                           & \multicolumn{1}{l|}{56,9 \%}                                                                                           \\ \hline
                                                                             &                                                                                                                               &                                                                                                                       &                                                                                                                        &                                                                                                                        \\ \hline
\multicolumn{1}{|l|}{30}                                                     & \multicolumn{1}{l|}{1}                                                                                                        & \multicolumn{1}{l|}{34,0 \%}                                                                                          & \multicolumn{1}{l|}{35,8 \%}                                                                                           & \multicolumn{1}{l|}{39,3 \%}                                                                                           \\ \hline
\multicolumn{1}{|l|}{30}                                                     & \multicolumn{1}{l|}{5}                                                                                                        & \multicolumn{1}{l|}{40,9 \%}                                                                                          & \multicolumn{1}{l|}{47,4 \%}                                                                                           & \multicolumn{1}{l|}{49,9 \%}                                                                                           \\ \hline
\multicolumn{1}{|l|}{30}                                                     & \multicolumn{1}{l|}{10}                                                                                                       & \multicolumn{1}{l|}{46,3 \%}                                                                                          & \multicolumn{1}{l|}{49,5 \%}                                                                                           & \multicolumn{1}{l|}{54,1 \%}                                                                                           \\ \hline
\multicolumn{1}{|l|}{30}                                                     & \multicolumn{1}{l|}{15}                                                                                                       & \multicolumn{1}{l|}{46,9 \%}                                                                                          & \multicolumn{1}{l|}{52,1 \%}                                                                                           & \multicolumn{1}{l|}{56,9 \%}                                                                                           \\ \hline
\multicolumn{1}{|l|}{30}                                                     & \multicolumn{1}{l|}{20}                                                                                                       & \multicolumn{1}{l|}{46,9 \%}                                                                                          & \multicolumn{1}{l|}{52,6 \%}                                                                                           & \multicolumn{1}{l|}{58,4 \%}                                                                                           \\ \hline
\multicolumn{1}{|l|}{30}                                                     & \multicolumn{1}{l|}{30}                                                                                                       & \multicolumn{1}{l|}{45,7 \%}                                                                                          & \multicolumn{1}{l|}{55,3 \%}                                                                                           & \multicolumn{1}{l|}{58,5 \%}                                                                                           \\ \hline
                                                                             &                                                                                                                               &                                                                                                                       &                                                                                                                        &                                                                                                                        \\ \hline
\multicolumn{1}{|l|}{50}                                                     & \multicolumn{1}{l|}{1}                                                                                                        & \multicolumn{1}{l|}{41,0 \%}                                                                                          & \multicolumn{1}{l|}{37,8 \%}                                                                                           & \multicolumn{1}{l|}{35,5 \%}                                                                                           \\ \hline
\multicolumn{1}{|l|}{50}                                                     & \multicolumn{1}{l|}{5}                                                                                                        & \multicolumn{1}{l|}{43,7 \%}                                                                                          & \multicolumn{1}{l|}{49,8 \%}                                                                                           & \multicolumn{1}{l|}{50,1 \%}                                                                                           \\ \hline
\multicolumn{1}{|l|}{50}                                                     & \multicolumn{1}{l|}{10}                                                                                                       & \multicolumn{1}{l|}{47,7 \%}                                                                                          & \multicolumn{1}{l|}{52,5 \%}                                                                                           & \multicolumn{1}{l|}{56,5 \%}                                                                                           \\ \hline
\multicolumn{1}{|l|}{50}                                                     & \multicolumn{1}{l|}{15}                                                                                                       & \multicolumn{1}{l|}{49,0 \%}                                                                                          & \multicolumn{1}{l|}{52,4 \%}                                                                                           & \multicolumn{1}{l|}{56,5 \%}                                                                                           \\ \hline
\multicolumn{1}{|l|}{50}                                                     & \multicolumn{1}{l|}{20}                                                                                                       & \multicolumn{1}{l|}{48,5 \%}                                                                                          & \multicolumn{1}{l|}{53,0 \%}                                                                                           & \multicolumn{1}{l|}{60,1 \%}                                                                                           \\ \hline
\multicolumn{1}{|l|}{50}                                                     & \multicolumn{1}{l|}{30}                                                                                                       & \multicolumn{1}{l|}{46,1 \%}                                                                                          & \multicolumn{1}{l|}{56,3 \%}                                                                                           & \multicolumn{1}{l|}{58,4 \%}                                                                                           \\ \hline
\end{tabular}
\caption{CIFAR-10 Modell, DPSGD mit Clipping bei 1,0 und ohne Augmentation}
\label{tab:c10_exp2}
\end{table}
\begin{table}[!htb]
\centering
\begin{tabular}{llllll}
\hline
\rowcolor[HTML]{EFEFEF} 
\multicolumn{6}{|l|}{\cellcolor[HTML]{EFEFEF}CIFAR-10 Modell mit Delta=$10^{-5}$ und Clipping bei 0,0001}  \\ \hline
\rowcolor[HTML]{CBCEFB} 
\multicolumn{1}{|l|}{\cellcolor[HTML]{CBCEFB}{\color[HTML]{000000} Epsilon}} & \multicolumn{1}{l|}{\cellcolor[HTML]{CBCEFB}{\color[HTML]{000000} \begin{tabular}[c]{@{}l@{}}Anzahl \\ Epochen\end{tabular}}} & \multicolumn{1}{l|}{\cellcolor[HTML]{CBCEFB}\begin{tabular}[c]{@{}l@{}}Genauigkeit \\ Batch-Größe\\ 64\end{tabular}} & \multicolumn{1}{l|}{\cellcolor[HTML]{CBCEFB}\begin{tabular}[c]{@{}l@{}}Genauigkeit\\ Batch-Größe\\ 128\end{tabular}} & \multicolumn{1}{l|}{\cellcolor[HTML]{CBCEFB}\begin{tabular}[c]{@{}l@{}}Genauigkeit\\ Batch-Größe\\ 256\end{tabular}} & \multicolumn{1}{l|}{\cellcolor[HTML]{CBCEFB}\begin{tabular}[c]{@{}l@{}}Genauigkeit\\ Batch-Größe\\ 512\end{tabular}} \\ \hline
\multicolumn{1}{|l|}{1}   & \multicolumn{1}{l|}{1} & \multicolumn{1}{l|}{26,8 \%} & \multicolumn{1}{l|}{30,1 \%} & \multicolumn{1}{l|}{30,9 \%} & \multicolumn{1}{l|}{30,3 \%} \\ \hline
\multicolumn{1}{|l|}{1}   & \multicolumn{1}{l|}{5} & \multicolumn{1}{l|}{32,3 \%} & \multicolumn{1}{l|}{37,4 \%} & \multicolumn{1}{l|}{42,2 \%} & \multicolumn{1}{l|}{42,7 \%} \\ \hline
\multicolumn{1}{|l|}{1}   & \multicolumn{1}{l|}{10}  & \multicolumn{1}{l|}{34,3 \%} & \multicolumn{1}{l|}{40,0 \%} & \multicolumn{1}{l|}{42,5 \%} & \multicolumn{1}{l|}{46,0 \%} \\ \hline
\multicolumn{1}{|l|}{1}   & \multicolumn{1}{l|}{15}  & \multicolumn{1}{l|}{34,1 \%} & \multicolumn{1}{l|}{41,1 \%} & \multicolumn{1}{l|}{44,1 \%} & \multicolumn{1}{l|}{46,3 \%} \\ \hline
\multicolumn{1}{|l|}{1}   & \multicolumn{1}{l|}{20}  & \multicolumn{1}{l|}{29,2 \%} & \multicolumn{1}{l|}{41,1 \%} & \multicolumn{1}{l|}{43,9 \%} & \multicolumn{1}{l|}{47,0 \%} \\ \hline
\multicolumn{1}{|l|}{1}   & \multicolumn{1}{l|}{30}  & \multicolumn{1}{l|}{31,4 \%} & \multicolumn{1}{l|}{40,4 \%} & \multicolumn{1}{l|}{44,5 \%} & \multicolumn{1}{l|}{46,7 \%} \\ \hline
 &  & & & & \\ \hline
\multicolumn{1}{|l|}{5}   & \multicolumn{1}{l|}{1} & \multicolumn{1}{l|}{28,9 \%} & \multicolumn{1}{l|}{34,9 \%} & \multicolumn{1}{l|}{38,5 \%} & \multicolumn{1}{l|}{30,3 \%} \\ \hline
\multicolumn{1}{|l|}{5}   & \multicolumn{1}{l|}{5} & \multicolumn{1}{l|}{38,6 \%} & \multicolumn{1}{l|}{44,9 \%} & \multicolumn{1}{l|}{45,9 \%} & \multicolumn{1}{l|}{48,4 \%} \\ \hline
\multicolumn{1}{|l|}{5}   & \multicolumn{1}{l|}{10}  & \multicolumn{1}{l|}{41,1 \%} & \multicolumn{1}{l|}{47,0 \%} & \multicolumn{1}{l|}{47,8 \%} & \multicolumn{1}{l|}{50,4 \%} \\ \hline
\multicolumn{1}{|l|}{5}   & \multicolumn{1}{l|}{15}  & \multicolumn{1}{l|}{40,3 \%} & \multicolumn{1}{l|}{46,0 \%} & \multicolumn{1}{l|}{49,4 \%} & \multicolumn{1}{l|}{53,7 \%} \\ \hline
\multicolumn{1}{|l|}{5}   & \multicolumn{1}{l|}{20}  & \multicolumn{1}{l|}{37,0 \%} & \multicolumn{1}{l|}{46,8 \%} & \multicolumn{1}{l|}{53,6 \%} & \multicolumn{1}{l|}{53,7 \%} \\ \hline
\multicolumn{1}{|l|}{5}   & \multicolumn{1}{l|}{30}  & \multicolumn{1}{l|}{41,7 \%} & \multicolumn{1}{l|}{50,2 \%} & \multicolumn{1}{l|}{54,8 \%} & \multicolumn{1}{l|}{55,0 \%} \\ \hline
 &  & & & & \\ \hline
\multicolumn{1}{|l|}{10}  & \multicolumn{1}{l|}{1} & \multicolumn{1}{l|}{31,3 \%} & \multicolumn{1}{l|}{31,0 \%} & \multicolumn{1}{l|}{33,4 \%} & \multicolumn{1}{l|}{20,4 \%} \\ \hline
\multicolumn{1}{|l|}{10}  & \multicolumn{1}{l|}{5} & \multicolumn{1}{l|}{41,6 \%} & \multicolumn{1}{l|}{43,0 \%} & \multicolumn{1}{l|}{46,9 \%} & \multicolumn{1}{l|}{47,2 \%} \\ \hline
\multicolumn{1}{|l|}{10}  & \multicolumn{1}{l|}{10}  & \multicolumn{1}{l|}{44,4 \%} & \multicolumn{1}{l|}{49,3 \%} & \multicolumn{1}{l|}{50,9 \%} & \multicolumn{1}{l|}{54,5 \%} \\ \hline
\multicolumn{1}{|l|}{10}  & \multicolumn{1}{l|}{15}  & \multicolumn{1}{l|}{42,3 \%} & \multicolumn{1}{l|}{48,6 \%} & \multicolumn{1}{l|}{52,8 \%} & \multicolumn{1}{l|}{54,4 \%} \\ \hline
\multicolumn{1}{|l|}{10}  & \multicolumn{1}{l|}{20}  & \multicolumn{1}{l|}{47,0 \%} & \multicolumn{1}{l|}{52,9 \%} & \multicolumn{1}{l|}{55,3 \%} & \multicolumn{1}{l|}{54,9 \%} \\ \hline
\multicolumn{1}{|l|}{10}  & \multicolumn{1}{l|}{30}  & \multicolumn{1}{l|}{45,5 \%} & \multicolumn{1}{l|}{52,1 \%} & \multicolumn{1}{l|}{55,7 \%} & \multicolumn{1}{l|}{59,4 \%} \\ \hline
 &  & & & & \\ \hline
\multicolumn{1}{|l|}{20}  & \multicolumn{1}{l|}{1} & \multicolumn{1}{l|}{33,0 \%} & \multicolumn{1}{l|}{37,7 \%} & \multicolumn{1}{l|}{35,1 \%} & \multicolumn{1}{l|}{25,9 \%} \\ \hline
\multicolumn{1}{|l|}{20}  & \multicolumn{1}{l|}{5} & \multicolumn{1}{l|}{44,6 \%} & \multicolumn{1}{l|}{46,9 \%} & \multicolumn{1}{l|}{49,5 \%} & \multicolumn{1}{l|}{49,2 \%} \\ \hline
\multicolumn{1}{|l|}{20}  & \multicolumn{1}{l|}{10}  & \multicolumn{1}{l|}{46,5 \%} & \multicolumn{1}{l|}{50,5 \%} & \multicolumn{1}{l|}{52,0 \%} & \multicolumn{1}{l|}{56,4 \%} \\ \hline
\multicolumn{1}{|l|}{20}  & \multicolumn{1}{l|}{15}  & \multicolumn{1}{l|}{45,0 \%} & \multicolumn{1}{l|}{51,1 \%} & \multicolumn{1}{l|}{55,8 \%} & \multicolumn{1}{l|}{57,7 \%} \\ \hline
\multicolumn{1}{|l|}{20}  & \multicolumn{1}{l|}{20}  & \multicolumn{1}{l|}{47,4 \%} & \multicolumn{1}{l|}{55,8 \%} & \multicolumn{1}{l|}{58,2 \%} & \multicolumn{1}{l|}{58,0 \%} \\ \hline
\multicolumn{1}{|l|}{20}  & \multicolumn{1}{l|}{30}  & \multicolumn{1}{l|}{50,6 \%} & \multicolumn{1}{l|}{56,3 \%} & \multicolumn{1}{l|}{58,0 \%} & \multicolumn{1}{l|}{60,5 \%} \\ \hline
 &  & & & & \\ \hline
\multicolumn{1}{|l|}{30}  & \multicolumn{1}{l|}{1} & \multicolumn{1}{l|}{37,0 \%} & \multicolumn{1}{l|}{36,4 \%} & \multicolumn{1}{l|}{38,7 \%} & \multicolumn{1}{l|}{25,2 \%} \\ \hline
\multicolumn{1}{|l|}{30}  & \multicolumn{1}{l|}{5} & \multicolumn{1}{l|}{45,0 \%} & \multicolumn{1}{l|}{45,2 \%} & \multicolumn{1}{l|}{51,5 \%} & \multicolumn{1}{l|}{49,6 \%} \\ \hline
\multicolumn{1}{|l|}{30}  & \multicolumn{1}{l|}{10}  & \multicolumn{1}{l|}{46,7 \%} & \multicolumn{1}{l|}{51,5 \%} & \multicolumn{1}{l|}{55,1 \%} & \multicolumn{1}{l|}{55,7 \%} \\ \hline
\multicolumn{1}{|l|}{30}  & \multicolumn{1}{l|}{15}  & \multicolumn{1}{l|}{49,1 \%} & \multicolumn{1}{l|}{53,3 \%} & \multicolumn{1}{l|}{55,4 \%} & \multicolumn{1}{l|}{59,1 \%} \\ \hline
\multicolumn{1}{|l|}{30}  & \multicolumn{1}{l|}{20}  & \multicolumn{1}{l|}{50,6 \%} & \multicolumn{1}{l|}{54,4 \%} & \multicolumn{1}{l|}{58,1 \%} & \multicolumn{1}{l|}{59,6 \%} \\ \hline
\multicolumn{1}{|l|}{30}  & \multicolumn{1}{l|}{30}  & \multicolumn{1}{l|}{51,2 \%} & \multicolumn{1}{l|}{56,9 \%} & \multicolumn{1}{l|}{59,9 \%} & \multicolumn{1}{l|}{61,9 \%} \\ \hline
 &  & & & & \\ \hline
\multicolumn{1}{|l|}{50}  & \multicolumn{1}{l|}{1} & \multicolumn{1}{l|}{36,6 \%} & \multicolumn{1}{l|}{40,5 \%} & \multicolumn{1}{l|}{38,6 \%} & \multicolumn{1}{l|}{28,8 \%} \\ \hline
\multicolumn{1}{|l|}{50}  & \multicolumn{1}{l|}{5} & \multicolumn{1}{l|}{45,1 \%} & \multicolumn{1}{l|}{49,1 \%} & \multicolumn{1}{l|}{51,3 \%} & \multicolumn{1}{l|}{51,3 \%} \\ \hline
\multicolumn{1}{|l|}{50}  & \multicolumn{1}{l|}{10}  & \multicolumn{1}{l|}{48,7 \%} & \multicolumn{1}{l|}{52,1 \%} & \multicolumn{1}{l|}{56,3 \%} & \multicolumn{1}{l|}{57,0 \%} \\ \hline
\multicolumn{1}{|l|}{50}  & \multicolumn{1}{l|}{15}  & \multicolumn{1}{l|}{49,2 \%} & \multicolumn{1}{l|}{54,8 \%} & \multicolumn{1}{l|}{58,6 \%} & \multicolumn{1}{l|}{60,4 \%} \\ \hline
\multicolumn{1}{|l|}{50}  & \multicolumn{1}{l|}{20}  & \multicolumn{1}{l|}{53,4 \%} & \multicolumn{1}{l|}{55,7 \%} & \multicolumn{1}{l|}{57,4 \%} & \multicolumn{1}{l|}{60,5 \%} \\ \hline
\multicolumn{1}{|l|}{50}  & \multicolumn{1}{l|}{30}  & \multicolumn{1}{l|}{53,7 \%} & \multicolumn{1}{l|}{56,7 \%} & \multicolumn{1}{l|}{60,0 \%} & \multicolumn{1}{l|}{63,2 \%} \\ \hline
\end{tabular}
\caption{CIFAR-10 Modell, DPSGD mit Clipping bei 0,0001 und ohne Augmentation}
\label{tab:c10_exp3}
\end{table}
\clearpage

\section{Hyperparametertuning DPSGD CelebA ResNet-18 Modell}
In Tabelle \ref{tab:r18_base} wird die Genauigkeit von zwei ResNet-18 Modellen gezeigt, wobei eines dieser Modelle zu Beginn vortrainiert war.
Dabei wurden die Daten über AutoAugment erweitert, was in Tabelle \ref{tab:r18_base2} jedoch entfernt wurde.
Tabelle \ref{tab:r18_exp1} zeigt, wie sich die Genauigkeit dieser Modelle bei Nutzung von DPSGD verändert. 
Hierbei sind jedoch noch keine Parameter optimiert.
Dies ist erst bei Tabelle \ref{tab:r18_exp2} der Fall, wo die virtuelle Batch-Größe erhöht ist und die Clipping-Norm verkleinert ist.
Jedoch ist Datenaugmentierung über AutoAugment zusätzlich integriert.
Tabelle \ref{tab:r18_vergleichAA} vergleicht, wie sich die Nutzung von Datenaugmentierung auf das Modell auswirkt.
\begin{table}[!htb]
\centering
\begin{tabular}{|l|l|l|l|}
\hline
\rowcolor[HTML]{CBCEFB} 
Epsilon  & \begin{tabular}[c]{@{}l@{}}Anzahl\\ Epochen\end{tabular} & \begin{tabular}[c]{@{}l@{}}vortrainiert,\\ mit AutoAugment\\ Batch-Größe=128\end{tabular} & \begin{tabular}[c]{@{}l@{}}nicht-vortrainiert,\\ mit AutoAugment\\ Batch-Größe=128\end{tabular} \\ \hline
$\infty$ & 1                                                        & 86,6 \%                                                                                   & 88,6 \%                                                                                         \\ \hline
$\infty$ & 3                                                        & 89,8 \%                                                                                   & 89,7 \%                                                                                         \\ \hline
$\infty$ & 5                                                        & 91,5 \%                                                                                   & 91,4 \%                                                                                         \\ \hline
$\infty$ & 10                                                       & 92,4 \%                                                                                   & 92,3 \%                                                                                         \\ \hline
\end{tabular}
\caption{Resnet-18 Modell mit AutoAugment ohne Differential Privacy}
\label{tab:r18_base}
\end{table}
\begin{table}[!htb]
\centering
\begin{tabular}{|l|l|l|l|}
\hline
\rowcolor[HTML]{CBCEFB} 
Epsilon  & \begin{tabular}[c]{@{}l@{}}Anzahl\\ Epochen\end{tabular} & \begin{tabular}[c]{@{}l@{}}vortrainiert,\\ ohne Augmentation\\ Batch-Größe=128\end{tabular} & \begin{tabular}[c]{@{}l@{}}nicht vortrainiert,\\ ohne Augmentation\\ Batch-Größe=128\end{tabular} \\ \hline
$\infty$ & 1                                                        & 87,1 \%                                                                                     & 86,0 \%                                                                                           \\ \hline
$\infty$ & 3                                                        & 89,9 \%                                                                                     & 90,7 \%                                                                                           \\ \hline
$\infty$ & 5                                                        & 91,1 \%                                                                                     & 90,8 \%                                                                                           \\ \hline
$\infty$ & 10                                                       & 91,0 \%                                                                                     & 90,1 \%                                                                                           \\ \hline
\end{tabular}
\caption{Resnet-18 Modell ohne Augmentation ohne Differential Privacy}
\label{tab:r18_base2}
\end{table}
\begin{table}[!htb]
\centering
\begin{tabular}{llll}
\hline
\rowcolor[HTML]{EFEFEF} 
\multicolumn{4}{|l|}{\cellcolor[HTML]{EFEFEF}\begin{tabular}[c]{@{}l@{}}CelebA ResNet-18-Modell mit Delta=$10^{-6}$, mit Augmentation, \\ Batch-Größe=64 und Clipping bei 1,0\end{tabular}}                                                                                                                     \\ \hline
\rowcolor[HTML]{CBCEFB} 
\multicolumn{1}{|l|}{\cellcolor[HTML]{CBCEFB}Epsilon} & \multicolumn{1}{l|}{\cellcolor[HTML]{CBCEFB}\begin{tabular}[c]{@{}l@{}}Anzahl \\ Epochen\end{tabular}} & \multicolumn{1}{l|}{\cellcolor[HTML]{CBCEFB}vortrainiert} & \multicolumn{1}{l|}{\cellcolor[HTML]{CBCEFB}nicht vortrainiert} \\ \hline
\multicolumn{1}{|l|}{1}                               & \multicolumn{1}{l|}{1}                                                                                 & \multicolumn{1}{l|}{79,9 \%}                            & \multicolumn{1}{l|}{79,4 \%}                                  \\ \hline
\multicolumn{1}{|l|}{1}                               & \multicolumn{1}{l|}{3}                                                                                 & \multicolumn{1}{l|}{79,5 \%}                            & \multicolumn{1}{l|}{78,5 \%}                                  \\ \hline
\multicolumn{1}{|l|}{1}                               & \multicolumn{1}{l|}{5}                                                                                 & \multicolumn{1}{l|}{79,4 \%}                            & \multicolumn{1}{l|}{79,5 \%}                                  \\ \hline
\multicolumn{1}{|l|}{1}                               & \multicolumn{1}{l|}{10}                                                                                & \multicolumn{1}{l|}{79,4 \%}                            & \multicolumn{1}{l|}{79,3 \%}                                  \\ \hline
                                                      &                                                                                                        &                                                         &                                                               \\ \hline
\multicolumn{1}{|l|}{5}                               & \multicolumn{1}{l|}{1}                                                                                 & \multicolumn{1}{l|}{80,1 \%}                            & \multicolumn{1}{l|}{80,2 \%}                                  \\ \hline
\multicolumn{1}{|l|}{5}                               & \multicolumn{1}{l|}{3}                                                                                 & \multicolumn{1}{l|}{80,5 \%}                            & \multicolumn{1}{l|}{80,2 \%}                                  \\ \hline
\multicolumn{1}{|l|}{5}                               & \multicolumn{1}{l|}{5}                                                                                 & \multicolumn{1}{l|}{80,7 \%}                            & \multicolumn{1}{l|}{80,5 \%}                                  \\ \hline
\multicolumn{1}{|l|}{5}                               & \multicolumn{1}{l|}{10}                                                                                & \multicolumn{1}{l|}{79,7 \%}                            & \multicolumn{1}{l|}{80,0 \%}                                  \\ \hline
                                                      &                                                                                                        &                                                         &                                                               \\ \hline
\multicolumn{1}{|l|}{10}                              & \multicolumn{1}{l|}{1}                                                                                 & \multicolumn{1}{l|}{81,2 \%}                            & \multicolumn{1}{l|}{80,8 \%}                                  \\ \hline
\multicolumn{1}{|l|}{10}                              & \multicolumn{1}{l|}{3}                                                                                 & \multicolumn{1}{l|}{81,4 \%}                            & \multicolumn{1}{l|}{81,4 \%}                                  \\ \hline
\multicolumn{1}{|l|}{10}                              & \multicolumn{1}{l|}{5}                                                                                 & \multicolumn{1}{l|}{80,8 \%}                            & \multicolumn{1}{l|}{81,5 \%}                                  \\ \hline
\multicolumn{1}{|l|}{10}                              & \multicolumn{1}{l|}{10}                                                                                & \multicolumn{1}{l|}{80,9 \%}                            & \multicolumn{1}{l|}{81,3 \%}                                  \\ \hline
\end{tabular}
\caption{Resnet-18-Modell mit Augmentation, Batch-Größe=64 und Clipping bei 1,0}
\label{tab:r18_exp1}
\end{table}
\begin{table}[!htb]
\centering
\begin{tabular}{llll}
\hline
\rowcolor[HTML]{EFEFEF} 
\multicolumn{4}{|l|}{\cellcolor[HTML]{EFEFEF}\begin{tabular}[c]{@{}l@{}}CelebA ResNet-18-Modell mit Delta=$10^{-6}$, mit Augmentation,\\ physische Batch-Größe=64, virtuelle Batch-Größe=256 und Clipping bei $10^{-5}$\end{tabular}}                                                           \\ \hline
\rowcolor[HTML]{CBCEFB} 
\multicolumn{1}{|l|}{\cellcolor[HTML]{CBCEFB}Epsilon} & \multicolumn{1}{l|}{\cellcolor[HTML]{CBCEFB}\begin{tabular}[c]{@{}l@{}}Anzahl \\ Epochen\end{tabular}} & \multicolumn{1}{l|}{\cellcolor[HTML]{CBCEFB}vortrainiert} & \multicolumn{1}{l|}{\cellcolor[HTML]{CBCEFB}nicht vortrainiert} \\ \hline
\multicolumn{1}{|l|}{1}                               & \multicolumn{1}{l|}{1}                                                                                 & \multicolumn{1}{l|}{81,0 \%}                            & \multicolumn{1}{l|}{80,2\%}                                   \\ \hline
\multicolumn{1}{|l|}{1}                               & \multicolumn{1}{l|}{3}                                                                                 & \multicolumn{1}{l|}{82,0 \%}                            & \multicolumn{1}{l|}{81,8 \%}                                  \\ \hline
\multicolumn{1}{|l|}{1}                               & \multicolumn{1}{l|}{5}                                                                                 & \multicolumn{1}{l|}{81,1 \%}                            & \multicolumn{1}{l|}{82,0\%}                                   \\ \hline
\multicolumn{1}{|l|}{1}                               & \multicolumn{1}{l|}{10}                                                                                & \multicolumn{1}{l|}{82,2 \%}                            & \multicolumn{1}{l|}{82,1\%}                                   \\ \hline
                                                      &                                                                                                        &                                                         &                                                               \\ \hline
\multicolumn{1}{|l|}{5}                               & \multicolumn{1}{l|}{1}                                                                                 & \multicolumn{1}{l|}{82,7 \%}                            & \multicolumn{1}{l|}{81,3\%}                                   \\ \hline
\multicolumn{1}{|l|}{5}                               & \multicolumn{1}{l|}{3}                                                                                 & \multicolumn{1}{l|}{84,5 \%}                            & \multicolumn{1}{l|}{83,7 \%}                                  \\ \hline
\multicolumn{1}{|l|}{5}                               & \multicolumn{1}{l|}{5}                                                                                 & \multicolumn{1}{l|}{85,1 \%}                            & \multicolumn{1}{l|}{84,4\%}                                   \\ \hline
\multicolumn{1}{|l|}{5}                               & \multicolumn{1}{l|}{10}                                                                                & \multicolumn{1}{l|}{85,3\%}                             & \multicolumn{1}{l|}{84,4\%}                                   \\ \hline
                                                      &                                                                                                        &                                                         &                                                               \\ \hline
\multicolumn{1}{|l|}{10}                              & \multicolumn{1}{l|}{1}                                                                                 & \multicolumn{1}{l|}{82,1 \%}                            & \multicolumn{1}{l|}{82,1\%}                                   \\ \hline
\multicolumn{1}{|l|}{10}                              & \multicolumn{1}{l|}{3}                                                                                 & \multicolumn{1}{l|}{84,9 \%}                            & \multicolumn{1}{l|}{85,0 \%}                                  \\ \hline
\multicolumn{1}{|l|}{10}                              & \multicolumn{1}{l|}{5}                                                                                 & \multicolumn{1}{l|}{85,2\%}                             & \multicolumn{1}{l|}{84,3\%}                                   \\ \hline
\multicolumn{1}{|l|}{10}                              & \multicolumn{1}{l|}{10}                                                                                & \multicolumn{1}{l|}{85,7\%}                             & \multicolumn{1}{l|}{85,1\%}                                   \\ \hline
\end{tabular}
\caption{Resnet-18-Modell mit Augmentation, Batch-Größe=256 und Clipping bei $10^{-5}$}
\label{tab:r18_exp2}
\end{table}
\begin{table}[ht!]
\centering
\begin{tabular}{llll}
\hline
\rowcolor[HTML]{EFEFEF} 
\multicolumn{4}{|l|}{\cellcolor[HTML]{EFEFEF}\begin{tabular}[c]{@{}l@{}}CelebA ResNet-18-Modell mit Delta=10-6, nicht vortrainiert,\\ physische Batch-Größe=64, virtuelle Batch-Größe=256 und Clipping bei $10^{-5}$\end{tabular}}                                                                \\ \hline
\rowcolor[HTML]{CBCEFB} 
\multicolumn{1}{|l|}{\cellcolor[HTML]{CBCEFB}Epsilon} & \multicolumn{1}{l|}{\cellcolor[HTML]{CBCEFB}\begin{tabular}[c]{@{}l@{}}Anzahl \\ Epochen\end{tabular}} & \multicolumn{1}{l|}{\cellcolor[HTML]{CBCEFB}mit AutoAugments} & \multicolumn{1}{l|}{\cellcolor[HTML]{CBCEFB}ohne Augmentation} \\ \hline
\multicolumn{1}{|l|}{1}                               & \multicolumn{1}{l|}{3}                                                                                 & \multicolumn{1}{l|}{81,8 \%}                                  & \multicolumn{1}{l|}{83,6 \%}                                    \\ \hline
\multicolumn{1}{|l|}{1}                               & \multicolumn{1}{l|}{10}                                                                                & \multicolumn{1}{l|}{82,1 \%}                                  & \multicolumn{1}{l|}{83.6 \%}                                          \\ \hline
                                                      &                                                                                                        &                                                               &                                                                \\ \hline
\multicolumn{1}{|l|}{5}                               & \multicolumn{1}{l|}{3}                                                                                 & \multicolumn{1}{l|}{83,7 \%}                                  & \multicolumn{1}{l|}{85,0 \%}                                    \\ \hline
\multicolumn{1}{|l|}{5}                               & \multicolumn{1}{l|}{10}                                                                                & \multicolumn{1}{l|}{84,4\%}                                   & \multicolumn{1}{l|}{86,4 \%}                                          \\ \hline
                                                      &                                                                                                        &                                                               &                                                                \\ \hline
\multicolumn{1}{|l|}{10}                              & \multicolumn{1}{l|}{3}                                                                                 & \multicolumn{1}{l|}{85,0 \%}                                  & \multicolumn{1}{l|}{85,7 \%}                                    \\ \hline
\multicolumn{1}{|l|}{10}                              & \multicolumn{1}{l|}{10}                                                                                & \multicolumn{1}{l|}{85,1 \%}                                  & \multicolumn{1}{l|}{86,8 \%}                                          \\ \hline
\end{tabular}
\caption{Resnet-18-Modell mit und ohne Augmentation}
\label{tab:r18_vergleichAA}
\end{table}
\clearpage


\section{Hyperparametertuning DPSGD CelebA Vision Transformer Modell}
Tabelle \ref{tab:vit_baseAA} zeigt die Genauigkeit der Vision Transformer Modelle, ohne die Nutzung von DPSGD, jedoch mit Datenaugmentierung.
Die Datenaugmentierung ist in Tabelle \ref{tab:vit_base_noAA} jedoch deaktiviert.
Tabellen \ref{tab:vit_dpsgd1} zeigt, wie sich die Genauigkeit bei der Nutzung von DPSGD ohne Parameteroptimierung verändert. 
Die Batch-Größe ist dabei auf 16 gesetzt, da die nächstgrößere Zweierpotenz als Batch-Größe den Speicher der Grafikkarte überladen würde.
In Tabelle \ref{tab:vit_dpsgd2} wird deshalb eine virtuelle Batch-Größe von 128 genutzt. 
Außerdem wurde die Datenaugmentierung deaktiviert.
Tabelle \ref{tab:vit_dpsgd3} reduziert zusätzlich die Clipping-Norm auf einen Wert von $10^{-5}$.
\begin{table}[!htb]
\centering
\begin{tabular}{|l|l|l|l|}
\hline
\rowcolor[HTML]{CBCEFB} 
Epsilon  & \begin{tabular}[c]{@{}l@{}}Anzahl\\ Epochen\end{tabular} & \begin{tabular}[c]{@{}l@{}}vortrainiert,\\ mit AutoAugment\\ Batch-Größe=64\end{tabular} & \begin{tabular}[c]{@{}l@{}}nicht-vortrainiert,\\ mit AutoAugment\\ Batch-Größe=64\end{tabular} \\ \hline
$\infty$ & 1  & 80,0 \% & 80,0 \%  \\ \hline
$\infty$ & 3  & 80,8 \%  & 80,6 \% \\ \hline
$\infty$ & 5                                                        & 81,7 \%                                                                                  & 81,4 \%                                                                                        \\ \hline
$\infty$ & 10                                                       & 82,9 \%                                                                                  & 82,4 \%                                                                                        \\ \hline
\end{tabular}
\caption{Vision Transformer Modell mit AutoAugment ohne Differential Privacy}
\label{tab:vit_baseAA}
\end{table}
\begin{table}[!htb]
\centering
\begin{tabular}{|l|l|l|l|}
\hline
\rowcolor[HTML]{CBCEFB} 
Epsilon  & \begin{tabular}[c]{@{}l@{}}Anzahl\\ Epochen\end{tabular} & \begin{tabular}[c]{@{}l@{}}vortrainiert,\\ mit AutoAugment\\ Batch-Größe=64\end{tabular} & \begin{tabular}[c]{@{}l@{}}nicht-vortrainiert,\\ mit AutoAugment\\ Batch-Größe=64\end{tabular} \\ \hline
$\infty$ & 1                                                        & 80,8 \%                                                                                  & 79,1 \%                                                                                        \\ \hline
$\infty$ & 3                                                        & 81,3 \%                                                                                  & 80,2 \%                                                                                        \\ \hline
$\infty$ & 5                                                        & 81,8 \%                                                                                  & 81,0 \%                                                                                        \\ \hline
$\infty$ & 10                                                       & 83,0 \%                                                                                  & 82,3 \%                                                                                        \\ \hline
\end{tabular}
\caption{Vision Transformer Modell ohne Augmentation ohne Differential Privacy}
\label{tab:vit_base_noAA}
\end{table}
\begin{table}[!htb]
\centering
\begin{tabular}{|llll|}
\hline
\rowcolor[HTML]{EFEFEF} 
\multicolumn{4}{|l|}{\cellcolor[HTML]{EFEFEF}\begin{tabular}[c]{@{}l@{}}Celeb A Vision Transformer Modell\\ mit Delta=$10^{-6}$, mit Augmentation,\\ Batch-Größe=16 und Clipping bei 1,0\end{tabular}}                            \\ \hline
\rowcolor[HTML]{CBCEFB} 
\multicolumn{1}{|l|}{\cellcolor[HTML]{CBCEFB}Epsilon} & \multicolumn{1}{l|}{\cellcolor[HTML]{CBCEFB}\begin{tabular}[c]{@{}l@{}}Anzahl\\ Epochen\end{tabular}} & \multicolumn{1}{l|}{\cellcolor[HTML]{CBCEFB}vortrainiert} & nicht-vortrainiert \\ \hline
\multicolumn{1}{|l|}{1}                               & \multicolumn{1}{l|}{1}                                                                                & \multicolumn{1}{l|}{79,4 \%}                              & 78,6 \%            \\ \hline
\multicolumn{1}{|l|}{1}                               & \multicolumn{1}{l|}{3}                                                                                & \multicolumn{1}{l|}{79,1 \%}                              & 78,1 \%            \\ \hline
\multicolumn{1}{|l|}{1}                               & \multicolumn{1}{l|}{5}                                                                                & \multicolumn{1}{l|}{79,4 \%}                              & 78,8 \%            \\ \hline
\multicolumn{1}{|l|}{}                                & \multicolumn{1}{l|}{}                                                                                 & \multicolumn{1}{l|}{}                                     &                    \\ \hline
\multicolumn{1}{|l|}{5}                               & \multicolumn{1}{l|}{1}                                                                                & \multicolumn{1}{l|}{78,3 \%}                              & 78,6 \%            \\ \hline
\multicolumn{1}{|l|}{5}                               & \multicolumn{1}{l|}{3}                                                                                & \multicolumn{1}{l|}{78,1 \%}                              & 78,9 \%            \\ \hline
\multicolumn{1}{|l|}{5}                               & \multicolumn{1}{l|}{5}                                                                                & \multicolumn{1}{l|}{78,8 \%}                              & 78,7 \%            \\ \hline
\multicolumn{1}{|l|}{}                                & \multicolumn{1}{l|}{}                                                                                 & \multicolumn{1}{l|}{}                                     &                    \\ \hline
\multicolumn{1}{|l|}{10}                              & \multicolumn{1}{l|}{1}                                                                                & \multicolumn{1}{l|}{79,4 \%}                              & 79,6 \%            \\ \hline
\multicolumn{1}{|l|}{10}                              & \multicolumn{1}{l|}{3}                                                                                & \multicolumn{1}{l|}{79,8 \%}                              & 78,9 \%            \\ \hline
\multicolumn{1}{|l|}{10}                              & \multicolumn{1}{l|}{5}                                                                                & \multicolumn{1}{l|}{79,4 \%}                              & 79,4 \%            \\ \hline
\end{tabular}
\caption{Vision Transformer Modell mit Augmentation, Batch-Größe=16 und \\Clipping bei 1,0}
\label{tab:vit_dpsgd1}
\end{table}
\begin{table}[!htb]
\centering
\begin{tabular}{|llll|}
\hline
\rowcolor[HTML]{EFEFEF} 
\multicolumn{4}{|l|}{\cellcolor[HTML]{EFEFEF}\begin{tabular}[c]{@{}l@{}}Celeb A Vision Transformer Modell\\ mit Delta=$10^{-5}$, ohne Augmentation,\\ virtuelle Batch-Größe=128 und Clipping bei 1,0\end{tabular}}                \\ \hline
\rowcolor[HTML]{CBCEFB} 
\multicolumn{1}{|l|}{\cellcolor[HTML]{CBCEFB}Epsilon} & \multicolumn{1}{l|}{\cellcolor[HTML]{CBCEFB}\begin{tabular}[c]{@{}l@{}}Anzahl\\ Epochen\end{tabular}} & \multicolumn{1}{l|}{\cellcolor[HTML]{CBCEFB}vortrainiert} & nicht-vortrainiert \\ \hline
\multicolumn{1}{|l|}{1}                               & \multicolumn{1}{l|}{1}                                                                                & \multicolumn{1}{l|}{79,4 \%}                              & 79,8 \%            \\ \hline
\multicolumn{1}{|l|}{1}                               & \multicolumn{1}{l|}{3}                                                                                & \multicolumn{1}{l|}{79,4 \%}                              & 80,0 \%            \\ \hline
\multicolumn{1}{|l|}{1}                               & \multicolumn{1}{l|}{5}                                                                                & \multicolumn{1}{l|}{79,5 \%}                              & 79,9 \%            \\ \hline
\multicolumn{1}{|l|}{}                                & \multicolumn{1}{l|}{}                                                                                 & \multicolumn{1}{l|}{}                                     &                    \\ \hline
\multicolumn{1}{|l|}{5}                               & \multicolumn{1}{l|}{1}                                                                                & \multicolumn{1}{l|}{79,4 \%}                              & 79,1 \%            \\ \hline
\multicolumn{1}{|l|}{5}                               & \multicolumn{1}{l|}{3}                                                                                & \multicolumn{1}{l|}{79,9 \%}                              & 79,4 \%            \\ \hline
\multicolumn{1}{|l|}{5}                               & \multicolumn{1}{l|}{5}                                                                                & \multicolumn{1}{l|}{79,9 \%}                              & 79,5 \%            \\ \hline
\multicolumn{1}{|l|}{}                                & \multicolumn{1}{l|}{}                                                                                 & \multicolumn{1}{l|}{}                                     &                    \\ \hline
\multicolumn{1}{|l|}{10}                              & \multicolumn{1}{l|}{1}                                                                                & \multicolumn{1}{l|}{79,6 \%}                              & 80,0 \%            \\ \hline
\multicolumn{1}{|l|}{10}                              & \multicolumn{1}{l|}{3}                                                                                & \multicolumn{1}{l|}{79,9 \%}                              & 78,6 \%            \\ \hline
\multicolumn{1}{|l|}{10}                              & \multicolumn{1}{l|}{5}                                                                                & \multicolumn{1}{l|}{80,1 \%}                              & 79,8 \%            \\ \hline
\end{tabular}
\caption{Vision Transformer Modell ohne Augmentation, virtuelle Batch-Größe=128 und Clipping bei 1,0}
\label{tab:vit_dpsgd2}
\end{table}
\begin{table}[!ht]
\centering
\begin{tabular}{|llll|}
\hline
\rowcolor[HTML]{EFEFEF} 
\multicolumn{4}{|l|}{\cellcolor[HTML]{EFEFEF}\begin{tabular}[c]{@{}l@{}}Celeb A Vision Transformer Modell\\ mit Delta=$10^{-6}$, ohne Augmentation,\\ virtuelle Batch-Größe=128 und Clipping bei $10^-5$\end{tabular}}                \\ \hline
\rowcolor[HTML]{CBCEFB} 
\multicolumn{1}{|l|}{\cellcolor[HTML]{CBCEFB}Epsilon} & \multicolumn{1}{l|}{\cellcolor[HTML]{CBCEFB}\begin{tabular}[c]{@{}l@{}}Anzahl\\ Epochen\end{tabular}} & \multicolumn{1}{l|}{\cellcolor[HTML]{CBCEFB}vortrainiert} & nicht-vortrainiert \\ \hline
\multicolumn{1}{|l|}{1}                               & \multicolumn{1}{l|}{1}                                                                                & \multicolumn{1}{l|}{80,1 \%}                              & 80,0 \%            \\ \hline
\multicolumn{1}{|l|}{1}                               & \multicolumn{1}{l|}{3}                                                                                & \multicolumn{1}{l|}{79,9 \%}                              & 80,1 \%            \\ \hline
\multicolumn{1}{|l|}{1}                               & \multicolumn{1}{l|}{5}                                                                                & \multicolumn{1}{l|}{80,0 \%}                              & 79,7 \%            \\ \hline
\multicolumn{1}{|l|}{}                                & \multicolumn{1}{l|}{}                                                                                 & \multicolumn{1}{l|}{}                                     &                    \\ \hline
\multicolumn{1}{|l|}{5}                               & \multicolumn{1}{l|}{1}                                                                                & \multicolumn{1}{l|}{79,3 \%}                              & 79,1 \%            \\ \hline
\multicolumn{1}{|l|}{5}                               & \multicolumn{1}{l|}{3}                                                                                & \multicolumn{1}{l|}{79,5 \%}                              & 79,2 \%            \\ \hline
\multicolumn{1}{|l|}{5}                               & \multicolumn{1}{l|}{5}                                                                                & \multicolumn{1}{l|}{80,0 \%}                              & 79,7 \%            \\ \hline
\multicolumn{1}{|l|}{}                                & \multicolumn{1}{l|}{}                                                                                 & \multicolumn{1}{l|}{}                                     &                    \\ \hline
\multicolumn{1}{|l|}{10}                              & \multicolumn{1}{l|}{1}                                                                                & \multicolumn{1}{l|}{79,8 \%}                              & 78,4 \%            \\ \hline
\multicolumn{1}{|l|}{10}                              & \multicolumn{1}{l|}{3}                                                                                & \multicolumn{1}{l|}{80,0 \%}                              & 79,9 \%            \\ \hline
\multicolumn{1}{|l|}{10}                              & \multicolumn{1}{l|}{5}                                                                                & \multicolumn{1}{l|}{80,1 \%}                              & 80,3 \%            \\ \hline
\end{tabular}
\caption{Vision Transformer Modell ohne Augmentation, virtuelle Batch-Größe=128 und Clipping bei $10^{-5}$}
\label{tab:vit_dpsgd3}
\end{table}
\clearpage

\section{Membership Inference Attacke CIFAR-10 Modell}
Tabelle \ref{tab:mi_cifar10_total} zeigt die Effektivität der Membership Inference Attacke gegen das CIFAR-10 Modell. 
Dabei werden unterschiedliche $\epsilon$-Werte, sowie eine steigende Anzahl an Shadow Modellen betrachtet
\begin{table}[!htb]
\centering
\begin{tabular}{lll}
\hline
\rowcolor[HTML]{CBCEFB} 
\multicolumn{1}{|l|}{\cellcolor[HTML]{CBCEFB}Epsilon} &
  \multicolumn{1}{l|}{\cellcolor[HTML]{CBCEFB}Anzahl Shadow Modelle} &
  \multicolumn{1}{l|}{\cellcolor[HTML]{CBCEFB}Genauigkeit des Angriff} \\ \hline
\multicolumn{1}{|l|}{$\infty$} & \multicolumn{1}{l|}{8}  & \multicolumn{1}{l|}{58,0 \%} \\ \hline
\multicolumn{1}{|l|}{$\infty$} & \multicolumn{1}{l|}{16} & \multicolumn{1}{l|}{58,9 \%} \\ \hline
\multicolumn{1}{|l|}{$\infty$} & \multicolumn{1}{l|}{32} & \multicolumn{1}{l|}{57,4 \%} \\ \hline
                               &                         &                              \\ \hline
\multicolumn{1}{|l|}{1}        & \multicolumn{1}{l|}{8}  & \multicolumn{1}{l|}{49,9 \%} \\ \hline
\multicolumn{1}{|l|}{1}        & \multicolumn{1}{l|}{16} & \multicolumn{1}{l|}{50,0 \%} \\ \hline
\multicolumn{1}{|l|}{1}        & \multicolumn{1}{l|}{32} & \multicolumn{1}{l|}{49,9 \%} \\ \hline
                               &                         &                              \\ \hline
\multicolumn{1}{|l|}{5}        & \multicolumn{1}{l|}{8}  & \multicolumn{1}{l|}{49,7 \%} \\ \hline
\multicolumn{1}{|l|}{5}        & \multicolumn{1}{l|}{16} & \multicolumn{1}{l|}{49,8 \%} \\ \hline
\multicolumn{1}{|l|}{5}        & \multicolumn{1}{l|}{32} & \multicolumn{1}{l|}{49,8 \%} \\ \hline
                               &                         &                              \\ \hline
\multicolumn{1}{|l|}{10}       & \multicolumn{1}{l|}{8}  & \multicolumn{1}{l|}{50,1 \%} \\ \hline
\multicolumn{1}{|l|}{10}       & \multicolumn{1}{l|}{16} & \multicolumn{1}{l|}{50,0 \%} \\ \hline
\multicolumn{1}{|l|}{10}       & \multicolumn{1}{l|}{32} & \multicolumn{1}{l|}{50,1 \%} \\ \hline
                               &                         &                              \\ \hline
\multicolumn{1}{|l|}{20}       & \multicolumn{1}{l|}{8}  & \multicolumn{1}{l|}{50,0 \%} \\ \hline
\multicolumn{1}{|l|}{20}       & \multicolumn{1}{l|}{16} & \multicolumn{1}{l|}{50,2 \%} \\ \hline
\multicolumn{1}{|l|}{20}       & \multicolumn{1}{l|}{32} & \multicolumn{1}{l|}{50,2 \%} \\ \hline
                               &                         &                              \\ \hline
\multicolumn{1}{|l|}{30}       & \multicolumn{1}{l|}{8}  & \multicolumn{1}{l|}{50,3 \%} \\ \hline
\multicolumn{1}{|l|}{30}       & \multicolumn{1}{l|}{16} & \multicolumn{1}{l|}{49,9 \%} \\ \hline
\multicolumn{1}{|l|}{30}       & \multicolumn{1}{l|}{32} & \multicolumn{1}{l|}{50,2 \%} \\ \hline
                               &                         &                              \\ \hline
\multicolumn{1}{|l|}{50}       & \multicolumn{1}{l|}{8}  & \multicolumn{1}{l|}{50,0 \%} \\ \hline
\multicolumn{1}{|l|}{50}       & \multicolumn{1}{l|}{16} & \multicolumn{1}{l|}{50,2 \%} \\ \hline
\multicolumn{1}{|l|}{50}       & \multicolumn{1}{l|}{32} & \multicolumn{1}{l|}{50,0 \%} \\ \hline
\end{tabular}
\caption{Membership Inference Angriff gegen CIFAR-10 Modelle}
\label{tab:mi_cifar10_total}
\end{table}